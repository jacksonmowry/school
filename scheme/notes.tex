% Created 2022-01-31 Mon 13:42
% Intended LaTeX compiler: pdflatex
\documentclass[11pt]{article}
\usepackage[utf8]{inputenc}
\usepackage[T1]{fontenc}
\usepackage{graphicx}
\usepackage{longtable}
\usepackage{wrapfig}
\usepackage{rotating}
\usepackage[normalem]{ulem}
\usepackage{amsmath}
\usepackage{amssymb}
\usepackage{capt-of}
\usepackage{hyperref}
\author{Jackson}
\date{\today}
\title{Scheme 101}
\hypersetup{
 pdfauthor={Jackson},
 pdftitle={Scheme 101},
 pdfkeywords={},
 pdfsubject={},
 pdfcreator={Emacs 27.2 (Org mode 9.6)}, 
 pdflang={English}}
\begin{document}

\maketitle
\tableofcontents


\section{Basic functions}
\label{sec:orge00c60c}
\begin{itemize}
\item (+ 1 2 3)
\item =6
\item Functions are a list where the first elements in each list ``()'' is a function
\item So for the first example ``+'' is the function we are calling, and ``1'', ``2'', ``3'' are our arguments
\item Some examples of basic operators are ``+, -, /, *, etc''
\end{itemize}
\subsection{Using multiple functions inline}
\label{sec:org19ae01d}
\begin{itemize}
\item Since everything is Scheme is a list we can call functions inside of functions, and these inner functions evaluate to a return value that is then passed to the next function out
\item (sqrt (+ (* 5 5) (* 12 12)))
\item 13
\item Let's break this example down
\begin{itemize}
\item There are two ways to think about this single line:
\begin{itemize}
\item Inside-out, and Outside-in
\end{itemize}
\item Let's start with Inside-out as this will help us view the function left-to-right
\begin{enumerate}
\item (sqrt \ldots{} )
\begin{itemize}
\item sqrt takes an argument and returns its square root
\end{itemize}
\item (+ \ldots{} )
\begin{itemize}
\item + is the addition opperator and it will add all arguments together
\end{itemize}
\item (* \ldots{} )
\begin{itemize}
\item * is the multiplication opperator that will multiply all arguments together and return the output
\end{itemize}
\end{enumerate}
\item So our line of code will return the square root of the result of the addition of the product of (5*5) and (12*12)
\begin{itemize}
\item Seems complicated when you look at it that way, so lets take a look from the inside out
\end{itemize}
\item From the inside-out we can start by finding the inner-most argument, which in out case is the second multiplication opperator
\begin{enumerate}
\item (* 12 12)
\begin{itemize}
\item = 144
\end{itemize}
\item Next we can perform next operation outwards,  (* 5 5)
\begin{itemize}
\item = 25
\end{itemize}
\item Now that we have performed both of the operations inside the inntermost function we can perform that function, (+ 25 144)
\begin{itemize}
\item = 169
\end{itemize}
\item So then we move outward and we can see we are at the outermost operation being sqrt, sqrt take exactly one argument being our number 169, (sqrt 169)
\begin{itemize}
\item = 13
\end{itemize}
\end{enumerate}
\end{itemize}
\end{itemize}
\begin{verbatim}
(sqrt (+ (* 5 5) (* 5 5)))
\end{verbatim}

\subsubsection{Side Note on Creating a List}
\label{sec:org6235270}
\begin{itemize}
\item Since Scheme uses lists for both function and data we need a way to tell it which is which. For this we prefix our list with a singlr quote '()
\item Notice that we only use a beginning quote, and there is no need for an ending quote mark.
\end{itemize}
\begin{verbatim}
(* 4 5)
= 20

'(* 4 5)
=(* 4 5)
\end{verbatim}
\end{document}
