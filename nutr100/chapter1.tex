% Created 2022-02-04 Fri 17:46
% Intended LaTeX compiler: pdflatex
\documentclass[letterpaper, 11pt]{article}
                      \usepackage{lmodern} % Ensures we have the right font
\usepackage[T1]{fontenc}
\usepackage[utf8]{inputenc}
\usepackage{graphicx}
\usepackage{amsmath, amsthm, amssymb}
\usepackage[table, xcdraw]{xcolor}
\definecolor{bblue}{HTML}{0645AD}
\usepackage[colorlinks]{hyperref}
\hypersetup{colorlinks, linkcolor=blue, urlcolor=bblue}
\usepackage{titling}
\setlength{\droptitle}{-6em}
\setlength{\parindent}{0pt}
\setlength{\parskip}{1em}
\usepackage[stretch=10]{microtype}
\usepackage{hyphenat}
\usepackage{ragged2e}
\usepackage{subfig} % Subfigures (not needed in Org I think)
\usepackage{hyperref} % Links
\usepackage{listings} % Code highlighting
\usepackage[top=1in, bottom=1.25in, left=1.55in, right=1.55in]{geometry}
\renewcommand{\baselinestretch}{1.15}
\usepackage[explicit]{titlesec}
\pretitle{\begin{center}\fontsize{20pt}{20pt}\selectfont}
\posttitle{\par\end{center}}
\preauthor{\begin{center}\vspace{-6bp}\fontsize{14pt}{14pt}\selectfont}
\postauthor{\par\end{center}\vspace{-25bp}}
\predate{\begin{center}\fontsize{12pt}{12pt}\selectfont}
\postdate{\par\end{center}\vspace{0em}}
\titlespacing\section{0pt}{5pt}{5pt} % left margin, space before section header, space after section header
\titlespacing\subsection{0pt}{5pt}{-2pt} % left margin, space before subsection header, space after subsection header
\titlespacing\subsubsection{0pt}{5pt}{-2pt} % left margin, space before subsection header, space after subsection header
\usepackage{enumitem}
\setlist{itemsep=-2pt} % or \setlist{noitemsep} to leave space around whole list
\usepackage{listings}
\author{Jackson Mowry}
\date{\textit{<2022-01-28 Fri>}}
\title{Nutrition for Your Life\\\medskip
\large Lee T. Murphy}
\hypersetup{
 pdfauthor={Jackson Mowry},
 pdftitle={Nutrition for Your Life},
 pdfkeywords={},
 pdfsubject={},
 pdfcreator={Emacs 27.2 (Org mode 9.6)}, 
 pdflang={English}}
\begin{document}

\maketitle
\tableofcontents


\section{Chapter 1: What We Eat and Why}
\label{sec:org0d13a05}
\subsection{Introduction}
\label{sec:orgfd00cd5}
The study of nutrition is ever-changing, and thus we are always discovering new ways that the foods we consume are connected to overall health and wellness. We need both quality and quantity of the right foods for optimum performance.\\
\subsubsection{Sciences Invloved in the Study of Nutrition}
\label{sec:org3ba3c04}
\begin{center}
\begin{tabular}{l}
Biochemistry\\
Biology\\
Physiology\\
Genetics\\
Psychology\\
Sociology\\
Anthropology\\
Immunology\\
Epidemiology\\
\end{tabular}
\end{center}
\subsection{Factors Affecting Food Choices}
\label{sec:org6137ef1}
There are many reasons we as humans choose to eat the foods we eat. \\
\textbf{Social Needs:}\\
Social events like a party or thanksgiving. Food is a social experience.\\
\textbf{Food Customs and Cultures:}\\
We don't get to choose our foods much as we grow up, so we end up eating the cultural norms of our parents.\\
\textbf{Food Cost:}\\
Looking for on-sale foods, or fast-food combo deals. Discounts like coupons or in-store deals will change where you choose to shop.\\
\textbf{Education and Occupation:}\\
If you are educatated on what foods are healthy you are more likely to choose those foods. Your lifestyle will affect how much time you have to plan and eat.\\
\textbf{Routines and Habits:}\\
Time of day will change what and where you can eat. Your schedule may make it convenient to eat out.\\
\textbf{Health Habits:}\\
Education on what you need to be healthy will play a role.\\
\textbf{Food Marketing:}\\
Check-out line purchases, colorful cereal boxes, tv ads. These all change what foods we decide to try or buy.\\
\textbf{Food Availability:}\\
You eat what is convenient to you, so if there are no healthy options you can't exactly just eat healthy.\\
\textbf{Taste and Flavor:}\\
We will eat what we enjoy the taste of.\\
\textbf{Psychological Needs:}\\
Our emotions will change our appetite, sad and we'll eat more, stressed and we'll go an entire afternoon without eating.\\
\textbf{Religion:}\\
Religious law or customs will restrict how people eat.\\
\textbf{Body Image:}\\
Some people eat or don't eat just to fit into their idea of a better body image.\\
\subsubsection{Supermarket Side-Track}
\label{sec:org0100681}
We don't really choose what we buy at the store. We end up buying things we didn't intend to buy. Our unconscious mind is swayed by food ads, and even without noticing it may influence us to buy food. Companies that pay more for shelf placement get their product on the middle shelf, right where we look when deciding what to buy. On top of this stores are set up like a labrynth, dark, and confusing.\\
We also have default product sizes, meaning we tend to gravitate towards the bigger sizes, and thus will eat more of the same food. Another factor influencing food choices is our baseline of any specific quality. If we eat a lot of salty food we will tend to purchase more salty foods to keep ourselves at that baseline. The food industry benefits from this default because they have been pumping more and more of this salt into foods. They also change our expectations by changing food pairings.\\
Lastly we have decision fatigue. The concept that making lots and lots of decisions over and over will eventually fatigue our brain causing us to give in to temptations much quicker. This plays out as we walk past end caps, and check-out stand displays right at the end of our shopping trip.\\
\subsection{What is Nutrition?}
\label{sec:org11393fd}
\begin{quote}
\emph{``The science of food, the nutrients and substances therin, their action, interaction and balance in relation to health \& disease, and the process by which the organism ingests, digests, absorbs, transports, and excretes food substances''} (The Council on Food and Nutrition of the American Medical Association)\\
\end{quote}
The primary reason for consuming food is to absorb their nutrients. These nutrients are vital for all bodily functions, called essential nutrients. If ommited you may experince bodily issues or abnormalities.\\
Diet and lifestyle cause 2/3 of the deaths in the US, while heart disease and cancer make up 1/2 of all deaths. Almost all of these diet/lifestyle deaths can be delayed or even prevented if better nutrition is introduced. Cancer is interesting because while some types can be caused by a poor diet, almost all types can have their effects lessened with proper diet and lifestyle.\\
\subsection{Obesity in the US}
\label{sec:org7528b4b}
What actually increases our risk of obesity or diabetes?\\

In the US 70 percent of adults are overweight or obese. In children the number is better at only 30 percent. Looking at how we got here reveals some interesting trends. Starting in 1985 the graph shows a max obesity percentage of 10-14, while in 1991 we see 15-19 percent, then in 1997 we see >20 percent. 2001 shows a >25 percent group, and >30 percent showing up in 2005. Now every state in America has more than 20 percent of its adult population classified as obese, with an additional 12 states reporting >35 percent or more. While this trend is just reporting obesity there is another interesting correlation (maybe even causation?): \textbf{diabetes}. The progression of diabetes has almost exactly followed the obesity trends, albeit at a lesser rate, peaking around 10 percent. Obesity is a preventable condition that if left untreated will lead to detrimental health outcomes.\\
\subsection{Six Classes of Nutrients}
\label{sec:orgeb119cf}
\begin{center}
\begin{tabular}{lll}
Name & Energy Yielding & Organic\\
\hline
Carbohydrates & Yes & Yes\\
Lipids & Yes & Yes\\
Proteins & Yes & Yes\\
Vitamins & No & Yes\\
Minerals & No & No\\
Water & No & No\\
\end{tabular}
\end{center}
From these nutrients we seek to abosorb what out body needs in order to go about our days. There are two categories of these nutrients \textbf{energy yielding} and \textbf{nonenergy yielding}. Energy yeilding nutrients help fuel our body in the form of calories (energy). The nonenergy yielding nutrients help with energy production, but do not themselves supply said energy.\\
\subsection{Macronutrients}
\label{sec:org47794b9}
This includes, carbs, fats, proteins, and water. They are measured in Calories/gram.\\
\textbf{Carbohydrates} can either be simple or complex. Simple sugars are small units of carbohydrates, while complex carbohydrates are made up of >2 simple sugars put together in many different forms. Another lesser form (not lesser in importance) of carbohydrates is Dietary Fiber. The body is not able to process this fiber, but instead it is fermented in the large intestine and its helps to promote a healthy GI tract. Plus this additional fiber seems to help lower the risk of colon cancer.\\
\textbf{Lipids (or fats)} are a dense energy store that can be called upon in times of fasting. Three common types are: tryglycerides, sterols, and phospholipids. Unsaturated fats are generally considered healthy while saturated fats are considered unhealthy.\\
\textbf{Protein} is the only macronutrient containing nitrogen, providing the body with amino acids. They make up almost all of our cells, and can sometimes be used directly (more so indirectly) for energy.\\
\textbf{Water} is a macronutrient due to the large quantity needed, although unlike the first three it does not provide energy. \textbf{9-13 cups a day} is reccomended from fluids and foods.\\
\subsection{Micronutrients}
\label{sec:org2740a6f}
This includes vitamins and minerals, called micro due to the relativly small amounts needed. They power and enable many chemical reactions in the body. These functions include: metabolism, growth and development.\\
\begin{center}
\begin{tabular}{ll}
Vitamin & Solubility\\
\hline
A & Fat\\
B & Water\\
C & Water\\
D & Fat\\
E & Fat\\
K & Fat\\
\end{tabular}
\end{center}
We also have minerals which help the nervous system, aid in water balance, support body structure, and other cellular processes.\\
\begin{center}
\begin{tabular}{ll}
Mineral & Quantity\\
\hline
Calcium & Major\\
Potassium & Major\\
Sodium & Major\\
Magnesium & Major\\
Iron & Trace\\
Zinc & Trace\\
Iodine & Trace\\
Chromium & Trace\\
Fluoride & Trace\\
\end{tabular}
\end{center}
\subsection{Other Constituents of Food (Nonnutritnets)}
\label{sec:org7f6b59c}
Nonnutrients are compounds in food not part of the previously addressed 6 nutrients. Herb, and spices fall on this list, along with phytochemicals, and polyphenols.\\
\subsection{Nutrition Math}
\label{sec:org03ee0d4}
Calories are a form of heat energy, we eat food and we produce heat. The first thing you should look at on a food lable is the portion size, all values and percentages are based on this amount.\\
\subsection{Recommendations for Macronutrients}
\label{sec:org6ecd201}
Different medical conditions will change how much of each nutrient we need.\\
\begin{center}
\begin{tabular}{rrrr}
Age & Carbohydrates & Protein & Fat\\
\hline
1-3 & 45-65 & 5-20 & 30-40\\
4-18 & 45-65 & 10-30 & 25-35\\
>19 & 45-65 & 10-35 & 20-35\\
\end{tabular}
\end{center}
These values are set by epideniologists that seek to reduse risk of chronic disease by sticking within these guidelines. These recommendations are not perfect, because they say nothing about the quality or type of food. We may consume the right percentage of carabohydrates but they mostly come from simple sugars. Likewise we may consume the right amount of protein and fat but they come mostly from animal sources leading to a higher intake of saturated fats.\\
\subsection{Improving Our Diets}
\label{sec:org1938984}
The \emph{Healthy People} initiative designs a 10 year plan to improve the health of all Americans. This includes thaking the \texttt{taking the initiative, the vision, mission, foundational principles, plan of action, and overarching goals}. The goals is to promote health while at the same time reducing the rusk for chronic disease. Eat healthy and maintain a healthy body weight. They seek to discover how to make a better environment that will be more condusive to these healthy behaviors.\\
\begin{quote}
\begin{itemize}
\item \emph{Consume a variety of nutrient-dense foods within and across the food groups, especially whole grains, fruits, vegetables, low-fat or fat-free milk or milk products, and lean meats and other protein sources.}\\
\item \emph{Limit the intake of saturated and trans fats, cholesterol, added sugars, sodium, and alcohol.}\\
\item \emph{Limit caloric intake to meet caloric needs and avoud unhealthy weight gain-including recommending that those whose weight is too high may also need to lose weight.}\\
\end{itemize}
\end{quote}
\subsection{Hunger, Appetite, and Satiety}
\label{sec:org9c1e041}
Hunger does not equal appetite. Hunger is the bodies need to eat to survive, and fuel our daily processes. Appetite is the mind telling you to eat, driven by our senses and environment. While these two are different things they do often occur together, though not always. What ties these two terms together is satiety: the feeling of fullness, or not needed to eat more to be satisfied.\\
\section{Chapter 2: What Am I Supposed To Consume? Recommendation for a Healthy Diet}
\label{sec:org69f5412}
\subsection{Introduction}
\label{sec:org525cd1b}
Often diet choices are confusing, you hear different things from just about every ``expert''. Building a healthy diet for you may just invlolve some simple swaps like substituting whole foods for more processed foods.\\
\subsection{Balance, Variety, and Moderation}
\label{sec:org1b14ab8}
Three buzz words of nutrition: variety, balanced, moderate. We must eat many different foods in order to get all of the proper nutrients. Look to eat from all the food groups, as well as eat many different foods from within each food group. Food color is an easy way to make sure that you are getting a variety of nutrients from the foods you are consuming.\\
We needs to make sure our diet is balanced, but what does this mean? Well we should seek to consume mostly healthy foods, with a little bit of fun foods sprinled in. You must also be cautious of how much you eat, this is an important factor to help maintain a healthy weight. Each person needs to eat a different amount of food to achieve their health goals. We try not to overconsume any one group of nutrients, as this may lead to hazards.\\
Another term is adequacy, meaning we want a diet that provides enough of each nutrient.\\
\subsubsection{Consume a VARIETY of foods, BALANCED by a MODERATE intake of each food}
\label{sec:org26ef952}
\emph{No one (natural) food supplies all the nuterint your body requires. Choosing a Variety of foods is essential to supply nutrients.}\\
\emph{Balance foods within and among food groups daily}\\
\emph{Stay physically active to Balance your eating patterns}\\
\emph{Control how much you eat through moderation}\\
\subsection{Nutrient Density versus Energy Density}
\label{sec:org98e64bd}
Nutrient density refers to the overall nutrient content of a food compared to its amount of calories (energy). Nutrient dense foods tends to be what we think of as healthy foods. Spinach is more nutrient dense than candy, and in the same manner a plain baked potato is more nutrient dense than its french fried coutner-part. We also have to look out for empty calores, things like sugar that contain almost no nutrients. They are simply energy (calories) that provide no nutrients to the body. You can easily choose higher nutrient dense foods by looking for the low-fat, or no added sugar versions of food.\\
Another important concept is energy density (or caloric density), which refers to a amount of energy in a food relative to its weight. Foods high in water and fiber, but low in fat and added sugars can often be considered non energy-dense foods. Desserts and sodas are energy-dense, while fresh fruits and vegetables are low in energy-density. Foods can be any conbination of the two factors. The foods we really want to avoid are those that are energy-dense and not nutrient-dense.\\
\subsection{How Do We Measure Nutritional Health?\ldots{} ABCDEs}
\label{sec:org6276d37}
\textbf{A}: Anthropometrics, or a persons physical measurements which can be compared to the healthy range for a persons age/height.\\
\textbf{B}: Biochemical,  nutrient quantity in the blood, urine, and feces.\\
\textbf{C}: Clinical, a physical exam looking over all aspects of your body. Some symptoms (and thus their causes) make themselves apparent, while others hide on the inside.\\
\textbf{D}: Dietary, a log of what a person eats every day, provifing patterns and behaviors.\\
\textbf{E}: Environmental, a look at a persons situation and how it impacts their food decision making power.\\
\subsection{States of Nutritional Health}
\label{sec:orge753a08}
Once we go over the ABCDEs we can create a picture of a persons nutritional health.\\
\textbf{Desirable} nutrition is defined as our energy/nutrient intale matching our bodies physical need. All body processess are able to function at their desired rate, and a persons body weight stays around the same value. simply: no excess, no deficiency.\\
\textbf{Malnutrition} is used to describe the condition of the body not being in the ideal nutritonal state. There are 2 forms on malnutrition:\\
\textbf{Undernutrition} is the state when are persons nutrient/energy intake is below what their body needs in order to function at its ideal state. This can manifest in the form of insufficient nutrient intake, or an overall lack of calories.\\
\textbf{Overnutrition} is the state of a persons intake exceeding their bodyily needs. This may lead to conditions like obesity, diabetes, and even toxic levels of some micronutrients.\\
\subsection{Brief History of Nutritional Guidance}
\label{sec:org16d3136}
The federal government creates recommendations based on research in order to promote healthy eating in the US. Over time these recommendations change leading to new guidelines that may change things that have been said in the past. We have gone from a wheel of 7 food groups, to 4 main food groups, then eventually back to 5. These every changing guidelines are a prime example the research never stops, and that when you ``fix'' one problem another problem might become more prevelent. The guideline now is called MyPlate, showing a graphic depiciton of a plate with 5 food groups seperated into section.\\
\subsection{MyPlate Overview}
\label{sec:org88ae2b9}
Veggies and fruit should fill half your plate, and be varied in color. A small glass of dairy indiciated that some dairy should be included in the diet, but not an excess. Grains should idealy come in the form of whole-grains, for fiber and nutrients. Lastly protein should come from lean sources, prepared in a manner that does not add additional fat.\\
\subsubsection{Limiatations of MyPlate}
\label{sec:org12bc6c5}
These guidelines are not recommended for children >2 years old. Children at this age need more fats as they are essential to their growth. Although the plate tries to show an abstracted ``American'' meal in doing so it eliminates any of the education surrounding individual food choices. Instead it shows colorful sections indicating exactly nothing, a viewer must take the time to go to the website or seek additional resources in order to find out what foods to consume.\\
\subsection{Dietary Guidelines For Americans}
\label{sec:org8274b96}
These guidelines seek to offer recommendations for what foods/nutrients should be consumed. Today overnutrition is becoming just as much of a concern as undernutrition. They seek to lower the risk of disease through recommendations. Recently there has been an increased focus to physical fitness being an important part of maintaining a healthy weight.\\
\begin{center}
\begin{tabular}{rl}
1 & Follow a healthy dietary pattern always\\
2 & Change and enjoy cultural foods to fit your lifestyle\\
3 & Focus on nutrient-dense foods to get your RDA\\
4 & Limit added sugars, saturated fats, sodium, and alcohol\\
\end{tabular}
\end{center}
It is important to eat healthy at all age ranges, and it is never to late to start eating heatlhy. Younger children need additional fats to suport growth, while adults can consume less fat to be fully functional. Food culture is important, so don't cut out your favorite foods all together, instead make chages to the recipe in order to make it fit in your nutritional goals. Staying within your calorie limit is very important, but of equal importance is getting all of your nutrients fufilled. So focus on nutrient dense foods to fill these macro and micronutrient goals. Limit anything mentioned in the 4th category above, the are detrimental to one's health. Additionally recommendations have been made to include some form of exercise in everyones lifestyle.\\
\subsection{Components to Reduce}
\label{sec:org5b3bb5f}
Although the federal government goes through all the trouble of putting out these recommendations more that 3/4 of the population fails to meet these guidelines, in one way or another. It seems like we love to overconsume calories, and underconsume the nutrient-dense foods we need.\\
\subsubsection{Added Sugars}
\label{sec:orgf082fec}
Those ``empty calories'' we mentioned early are very prevelent in soft drinks. Just having one soda will put you within 15 percent of your RDA for added sugars. Reducing one's consumption of added sugars will greatly reduce the risk of obesity, type 2 diabetes, and some cancers.\\
\subsubsection{Saturated Fat}
\label{sec:org3639a15}
Strive for less than 10\% of your daily calories coming from saturated fat, for heart health go for 5-6\%. Replacing saturated fat with unsaturated fat will lessen your risk for cardiovascular disease. Replacing saturated fats with carbohydrates does not reduce your risk of cardiovascular disease. There is no evidence that dietary cholesterol increases serum cholesterol, but the recommendations state that you should eat as little as possible.\\
\subsubsection{Sodium}
\label{sec:orgc341d0c}
Sodium intake should not exceed 2300 mg/day (about a teaspoon). For those at risk of certain heart conditions sodium intake is limited to 1500 mg/day.\\
\begin{center}
\begin{tabular}{l}
Populations at risk\\
\hline
Individuals over the age of 50\\
African Americans\\
Those with high bolld pressure\\
Those with diabetes\\
Those with chronic kidney disease\\
\end{tabular}
\end{center}
These at-risk groups make up about half of the US population. As sodium intake goes up so does blood pressure, this is a problem because many of the processed foods that make up the American diet contain excess sodium.\\
\subsubsection{Refined Grains}
\label{sec:org4a6971b}
These grains are the product of grains that have had their nutrient-dense outer later (bran) removed. In addition many of these grains will contain added sugars in their final forms.\\
\subsubsection{Table of Foods to Reduce}
\label{sec:org8ad126e}
\begin{center}
\begin{tabular}{ll}
Food Component to Reduce & Method to Reduce\\
\hline
Added Sugars & Consume less than 10\% of daily calories from added sugars\\
Saturated Fats & Consume less than 10\% of daily calories from saturated fats\\
Sodium & Consume less then 2300 mg/day (1500 mg/day for at-risk)\\
Alcohol & In moderation,  1 drink/day female,  2 drinks/day male\\
\end{tabular}
\end{center}
\subsection{Components to Increase}
\label{sec:orgd2da70f}
There are some foods we are currently under-consuming, it might be a good idea to look at some of these foods as potential replacements for the foods we are currently over-consuming. One group we are under-consuming is whole grains. Instead we are consuming refined grains at much higher quantites. Another group we generally under-consume is vegetables. We could also use some more seafood in our diets. Once again all of the evidence is pointing to the fact that we as Americans need more variety and balance in our diets.\\
\subsection{Dietary Guidelines and Your Food Choices}
\label{sec:org0fb1c61}
While many of the core recommendations about diet have remained the same over time, don't think that just becuase they don't change doesn't mean that they are incorrect. In fact much of the research has confirmed many of the old beliefs or recommendations, while some of the research has disagreed with past recommendations. Becuase of this food and diet guidelines change all the time, an ever evolving cycle of research and refinment.\\
There are also additional factors at play when viewing how people choose what they eat. For example individual likes and dislikes play a huge role in the ``everyday'' foods we eat. In addition someone who knows a ton about nutrition will make ``smarter'' choices than someone who is entierly uneducated about the subject. One must also consider the environmental influences a person faces day to day. If all of your co-workers eat out for lunch everyday you are more than likely to join them at least once. Your household may also play a role in what foods are convenient to eat, and what variety of food you have available to you.\\
Influence from marketing is more prevelent now than it ever has been before. The amount of ads we see just based on food is very high, and even if you are not consciously paying attention to the ad it may have a subconscious influence next time you are are out-and-about and find yourself in need of a meal.\\
Lastly it is important to consider social and cultual norms. For example how many parties or social-gatherings have you been to where there wasn't some sort of food/beverage being served. In fact we even have entire events centered around the coming together and sharing of food (Thanksgiving, Potlucks, Business Dinners. . . )\\
\subsubsection{Advicde from the Academy of Nutrition \& Dietetics}
\label{sec:orgc868782}
\begin{center}
\begin{tabular}{l}
Be realistic, make small changes\\
Be adventurous, try new foods\\
Be flexible, balance food with activity\\
Be active daily\\
\end{tabular}
\end{center}
\subsection{Consumer Food Guides: Dietary Reference Intakes}
\label{sec:org8a369fd}
The DRIs are set to help a person know how much of each nutrient they need for their individual person. This breaks down further into Estimated Average Requirements and Recommended Dietary Allowances.\\
Estimated Average Requirements refers to the amount of each nutrient a person should intake based on research for that nutrient. 50/50, half we need more, half will need less.\\
Recommended Dietary Allowances takes the EAR and bumps it up to 98/2, meaining 98 percent of the population will be good, without taking this number too high into the possible toxic levels.\\
Of course these above recommendations are very general as such individuals experiencing prolonged sickness or disease will need to consult their health care provider.\\
Another nutrient intake lable is Adequate Intake,  this is established when there is insufficient research to set an RDA, so researchers will set an AI range for that nutrient.\\
Next we have Tolerable Upper Intake Levels,  these limits are established for vitamins and minerals that are considered toxic if consumed in too high of quantities.\\
Lastly we have Daily Values, these are set on food lables so that we have a general idea of how this food will fit into our daily calorie allotment.\\
\subsubsection{How are DRIs Established?}
\label{sec:orgdff8dab}
\begin{center}
\begin{tabular}{ll}
Dietary Reference Intakes & Estimate the amount of a nutrient needed\\
Estimated Average Requirements & Amount of nutrient for 50\% of the population\\
Recommended Daily Allowance & The EAR moved up to 98\% of the population\\
Adequate Intake & Used when no EAR exists for the nutrient\\
\end{tabular}
\end{center}
\section{Chapter 3: Am I Consuming What My Body Needs?}
\label{sec:orgf7c41d2}
\subsection{Introduction}
\label{sec:orgde253f6}
It is once again important to remind ourselves that a varied and balanced diet is far more important than picking the ``one'' correct food to eat forever. Let's look at how we can take nutrient intake recommendations and apply them to our lives.\\
\subsection{In the Past: The Food Guide Pyramid}
\label{sec:orgddfc551}
The food pyramid was mainly used as a teaching tool with its ability to help an individual select components of a healthy diet. It also seeks to encourage each person to include some form of physical activity in their daily life. The size of pyramid sections is meant to indicate how much of each sections to consume.\\
\subsection{ChooseMyPlate}
\label{sec:org13bc821}
After recommendation from health professionals and consumers alike a new graphic was designed in 2011: MyPlate. Instead of representing food groups as slim triangles making up a pyramid this graphic choose to use a typical plate split into different food groups. 5 section split up into Fruits, Grains, Protien, Vegetables, and Dairy. The sizes go Vegetables > Fruits > Grains >  Proteins > and Dairy.\\
Some of the goals of the new graphic were: Balancing Calories, (Increasing Fruits, Vegetables, and Whole Grains), and reducing the less healthy subnutrients. The plate can help someone meet their nutrient recommendations while not exceeding their calorie allotment. The guide hopes to focus your food choices towards nutrient-dense foods to make up your standard nutrient allotment, then adding foods you would like to eat on top of that.\\
\subsection{In-Depth on the Food Groups}
\label{sec:org2d7e9aa}
\subsubsection{Grains}
\label{sec:org8b682d2}
In a 2000 calorie diet the recommendation would say that you need to eat 6 oz of grains, with 3 oz coming from whole grains. This recommendation seeks to increase whole grains and reduce refined grains. The fiber from whole grains can help with problems such as weight maintenance, digestion, cholesterol, and blood sugar control.\\
\subsubsection{Protein}
\label{sec:orgd0fb7a9}
As Americans we consume plenty of the main meat and nut based protein, but we lack in seafood protein. We should also look to increase the nutrient-density of our protein choices by picking the low-fat or lean options. Along with making type of meat swaps we can look to get our protein for otehr sources to help further lower saturated fat and cholesterol intake. For some (mainly males) need to reduce overall intake of protein in order to make room for more vegetables. Recommended amounts for a 2000 calorie diet are around 5.5 oz of meat.\\
\subsubsection{Vegetables}
\label{sec:org36a4afb}
There are 5 classes of vegetables: Dark Green, Orange, Starchy, Dry Beans and Peas, and Other. A person should seek to consume 2.5 cups of vegetables daily. The current guidelines would like to see individuals consume more vegetables, and a greater variety of vegetables. Look to consume a vegetable at every meal. If canned vegetables are to be consumed make sure minimal sodium is added during processing.\\
\subsubsection{Fruits}
\label{sec:orgc80154c}
Look to consume whole fruits, around 2 cups a day. All preperations of fruit are acceptable as long as they are in their pure form without any added sugars.\\
\subsubsection{More Matters and ``5-A-Day'' Goals for Fruits and Vegetables}
\label{sec:orgd25a1c2}
Eating your recommended quantity of fruit will help you maintain weight, stay healthy, and reduce the risk for chronic disease. 5-A-Day was added to raise awareness of thr benefits and needs of eating healthy food. Now they just recommend eating \emph{more} fruits and vegetables as opposed to the previous 5-9.\\
\subsubsection{Dairy}
\label{sec:org8de5ec3}
Reccomendations state that 3 cups of dairy should be consumed daily. Most individuals would benefit from an increase in consumption of low-fat or non-fat forms of daily products. Try to avoid cheese for its unecessary saturated fats.\\
\subsubsection{Oils and Fats}
\label{sec:org14a440e}
Oils and Fats are not even mentioned on the current version of myplate, meaning none should be consumed on their own. Instead a person should seek to get all of the daily fat from other sources in their unsaturated forms. Try to swap unsaturated fats for saturated fats in any place you can.\\
\subsubsection{Physical Activity}
\label{sec:orgbfede0b}
Physical activity is a firm recommendation for everyday life. For those ages 18-64 150 minutes of moderate intensity exercise is recommended.\\
\subsubsection{Beverages}
\label{sec:org360e7df}
Although some may not consider beverages when logging food intake, they can often add a large number of calories per beverage. Sugar sweetened beverages can easily add a ton of calories. Seek to consumer calorie free beverages, and nutrient-dense beverages (milk or 100\% juice).\\
\subsection{Energy Balance \& Healthy Weight}
\label{sec:orgadefa11}
Maintaining a healthy weight is a simple formula of energy in and energy out. Undereat and you will lose weight, overeat and you will gain weight. Adding 100 calories per day can add up over a year to a 10lb weight gain. As someone ages their need for food (and thus their metabolism) will decrease, meaning they need to eat less to maintain their current weight.\\
Here are some considerations when making choices regarding weight control\\
\begin{center}
\begin{tabular}{ll}
Control your portions & Know how much food you need to fill your calorie allowance\\
Control you environment & keep tempting foods our of the home, know when you can eat out\\
Avoid overeating & Make conscious choices about how much food you eat all days\\
Consider repackaging & Create smaller portions for yourself out of the entire meal\\
\end{tabular}
\end{center}
\subsection{Nutrition Information}
\label{sec:orgd62450c}
Nutrition facts are vital when making healthy diet choices, they allow us to know what and how much we are putting in our bodies.\\
\subsubsection{Serving Size/Servings per Container}
\label{sec:org0006865}
Look for serving size, along with servings per container. This information will help you relate the full container size with how much you can consume within your daily allowance. Serving sizes has recently been increased in order to more accuratly match how much people actually eat. Packaged foods will now have both a per serving, and per package nutriton facts.\\
\subsubsection{Calories}
\label{sec:org2fcfa5f}
Calories are now labeled in a bigger font for easily readibility, as caloreis are the most important factor in diet choices.\\
\subsubsection{The Nutrients}
\label{sec:org6b12512}
On the nutrition lable you will see certain key nutrients highlighted.\\
\begin{center}
\begin{tabular}{lll}
Nutrient & Limit & Increase\\
Saturated Fat & x & \\
Trans Fat & x & \\
Cholesterol & x & \\
Sodium & x & \\
Dietary Fiber &  & x\\
Vitamin D &  & x\\
Iron &  & x\\
Potassium &  & x\\
\end{tabular}
\end{center}
An interesting note is that calories from fat was removed after research showed that the type of fat consumed was much more important than overall fat consumed.\\
\subsubsection{The Percent Daily Value}
\label{sec:orgec3a737}
Percent Daily Value is based on a 2000 calorie diet and shows what percent of your daily allotment would be fulfulled by eating this food.\\
\subsubsection{Label Lingo: Food Label Terms and Health Claims}
\label{sec:org166b923}
When a nutrient has been found to have positive health benefits it will often be advertised or highlighted on food packaging.\\
A \textbf{Nutrient Content Claim} describes the level of a nutrient or dietary substance in the product (free, low, reduce, lite, more, and high).\\
\begin{center}
\begin{tabular}{ll}
Free & May only contain a trivial amount that has little physiological effects\\
Calorie Free & >5 calories per serving\\
Low Calorie & >=40 calories per serving\\
Less or Reduced Calories & 25\% fewer calories than the reference food\\
Fat free & Less than 0.5g fat per serving\\
Low fat & 3g fat per serving\\
Reduced or less Fat & 25\% less fat than the reference food\\
Saturated fat free & >0.5g saturated fat\\
Low in Saturated Fat & >1g and >15\% from saturated fat\\
0 Trans Fats & >0.5g\\
Sodium Free & Less than 5mg\\
Low Sodium & >140mg\\
Sugar Free & >0.5g sugar\\
No Added Sugars & No added sugar during processing\\
Good Source of & 10-19\% DV for that nutrient\\
High in & >20\% DV\\
\end{tabular}
\end{center}
\textbf{Health Claims} describe an ingredients link to reducing the risk of disease or other health condition. These claims must be vague. You essentially just need to say ``may'' reduce the risk of \uline{\uline{\_}}.\\
\textbf{Structure/Function} refers to claims about an ingredient boosting the bodys processes making a healthier human.\\
\subsection{Food Allergies and the Food Label}
\label{sec:orgccd6869}
There are a few common allergens that must be listed on food labels. Milk, eggs, peanuts, tree nuts, shellfish, soy, and wheat. Disclaimers like ``this product was manufactured in a plant that also processes wheat'' are voluntary.\\
\subsection{Organic Products and the Food Label}
\label{sec:org2f9cbdd}
While organic foods do not contain more nutrients within, they may reduce the chance of exposure to pesticide residues and antibiotic-resistant bacteria.\\
\begin{center}
\begin{tabular}{ll}
Cage-free & flock must be able to freely roam in an enclosed area, access to water and food\\
Free-range & flock is provided shelter and allowed to be outside\\
Natural & minimally processed meat/eggs\\
Grass-fed & animals recieve a majority of their nutrients through grass\\
\end{tabular}
\end{center}
\section{Chapter 4: How Do I Get Energy From My Food?}
\label{sec:org69d2285}
\subsection{Digestion and Absorption}
\label{sec:org7810a05}
All the foods we put into our body have a certain function, and in order to perform this function our body needs to be able to absorb the nutrients contained within. This is espicially important for those nutrients considered essential, meaning our body cannot produce them on its own.\\
\subsubsection{Body Systems and Nutrients}
\label{sec:orgc20ffc4}
Many complex organ systems must work together in order for our body to function. As nutrients come in they are used to create new substances, and the old substances are recycled.\\
The cell membrane is a bilayer of lipids, protiens, and cholesterol. It holds all the cell together and allows substances to pass in and out.\\
The cell nucleus is the componenet responsible for controlling cellular actions and directs protein synthesis and cell division.\\
\subsection{How Are Nutrients Transported?}
\label{sec:org8fbca72}
The four different types of collular absorption are: passove, facilitated, active, and phagocytosis/pinocytosis.\\
\subsubsection{Passive Diffusion}
\label{sec:org46e0831}
The process where lipids, water, and minerals are absorbed through what is essentially osmosis, no energy is required for the process.\\
\begin{center}
\includegraphics[width=.9\linewidth]{/home/jarch/Screenshots/maim-region-20220129-193008.png}
\end{center}
\subsubsection{Facilitated Diffusion}
\label{sec:org976303f}
High concentration nutrients move across the membrane with the packaging of a carrier protein. The common sugar fructose uses this method to move in to a cell without energy.\\
\begin{center}
\includegraphics[width=.9\linewidth]{/home/jarch/Screenshots/q.png}
\end{center}
\subsubsection{Active Transport}
\label{sec:orgf836391}
A transport method in which a carrier protein and energy are used to move the nutrient. Glucose, amino acids, and minerals use this method.\\
\subsubsection{Phagocytosis/Pinocytosis}
\label{sec:orgf054a5e}
These two processes invlove the cell engulfing the nutrient before being passed through the cell membrane. Phag is for solids, and Pino is for liquids. The lymph system, and the cardiovascular system also use these methods for transporting nutrients.\\
\begin{center}
\includegraphics[width=.9\linewidth]{/home/jarch/Screenshots/-region-20220129-194435.png}
\end{center}
\begin{center}
\includegraphics[width=.9\linewidth]{/home/jarch/Screenshots/-region-20220129-194550.png}
\end{center}
\subsubsection{Summary of the Movement of food through the Digestive System}
\label{sec:orgf616642}
Here is whats happends to the food we eat as it passed through our body. In our mouths food is broken down through a process called mastication, where food is combined with saliva. an enzyme called salivary amylase will further break down food after chewing. This food is then passed to the esophagus where the epiglottis makes sure good goes to your stomach and not your lungs. By this point the clump of food is called the bolus, which is moved to the stomach along with the saliva and enzymes. Once in the stomach HCl will continue to break down food killing mnay microorganisms. The stomach is responsible for breaking down protiens and fats, until is is ready to be passed on to the intentines. At this point the clump is called chyme, further enzymes are added to help get the remaining nutrients out. Additionally the liver adds a substance called bile to help coat fats, making sure that they cannot combine back into larger fat molecules. all usable nutrients are absorbed through the bloodstream, while the unusable materials pass into the colon, which is eventually passed as waste.\\
\subsubsection{Anatomy of The Digestive System}
\label{sec:orgea4c6c7}
There are two major componenets of the human digestive system: the \textbf{digestive tract} and the \textbf{accessory organs}. While the entire digestive tract can be thought of as one continuous tube, the accessory organs are external elements that aid in digestion.\\
\begin{center}
\begin{tabular}{ll}
Mouth & Teeth+Tongue=Mastication, Saliva+Enzymes=Chemical Digestion\\
Esophagus & Muscular tube tomove food via a involuntary mechanial process\\
Stomach & Food breakdown and storage before the intestines\\
Small Intestine & Enzymes+foods, breaks down and extracts nutrients>bloodsteam\\
Larger Intestine & Uses bacteria to ferment/breakdown waste\\
Rectum \& Anus & Stores feces before elimination from the body\\
\end{tabular}
\end{center}
The accessory organs aid the digestion process but are not directly involved in the process.\\
\begin{center}
\begin{tabular}{ll}
Pancreas & Produces digestive enzymes and alkaline fluid (pH neutralization)\\
Liver & Produces bile (coat fats), to mix fats with water\\
Gallbladder & Stores bile\\
\end{tabular}
\end{center}
\subsubsection{Mouth}
\label{sec:orge3ccb5f}
The mouth mainly serves to ingest food and perform the first large break down. By breaking food down into smaller more easily digested pieces the digestion process can take place even faster. Saliva serves two purposes in the mouth, lubrication, and adding enzymes to the bolus. Amylase is responsible for breaking down starches, while lipase is responsible for breaking down fat.\\
\subsubsection{Esophagus}
\label{sec:org75352ff}
The esophagus is the region where the peristalsis process takes place. This is an automatic process where food is moved down the esophagus towrds the stomach. The lower esophageal sphincter is responsible for passing food into the stomach, while at the same time preventing stomach acid from making its way out of the stomach.\\
\subsubsection{Stomach}
\label{sec:org578418c}
The average adult stomach has a size of about 1 liter, which holds a few different substances thorughout the day. Starting with chyme the name for food combined with gastric juices. These gastric juices are known as HCl an acidic substance that further breaks down food. This is such an acidic substance the the stomach lining needs to be strong enough to protect itself, the stomachs mucus shield serves this purpose. HCl will also active some enzymes to start working. While in the stomach food will be broken down for around 2-6 hours, the food is moved around and broken down until it reaches the desired consistency, at this point it will be passed into the intenstines by the pyloric sphincter.\\
\subsubsection{Small Intestine}
\label{sec:org17a07c9}
In the small intestine food is mixed with even more digestive enzymes to break down nutrients into their most simple parts. Then these nutrients are able to be absorbed into the blood stream though in high surface area of the small intestine. Within the small intestine there are many villi, these small fingerlike projections contain small hairlike cells which even further increase the surface area. With the muscles in the small intestine everything is in a wavelike motion allowing molecules to be trapped and easily absorbed.\\
The furst section of the small intestine is called the duodenum, a 12 inch long section responsible for the first absorption of food. Bile from the gallbladder enters the duodenum acting as an emulsifer which encapsulates fat moleucles allowing them to mix with water. The follwing sections of the small intestine are called the \textbf{jejumum} and the \textbf{ileum}.\\
\subsubsection{Large Intestine}
\label{sec:orgb2df962}
The large intestine (also known as the colon) is the area of the digestive tract where waste is prepared. By the time nutrients reach this portion all of the desired nutrients have been absorbed. Within the colon remainng water, sodium, potassium, and fatty acids are reabsorbed. Waste moves through the colon where various microorganisms continue to break it down, and more and more water is removed as it makes its way towards the end. These sections that waste moves through are called: the cecum, ascending colon, transverse colon, descending colon, and sigmoid colon.\\
The whole digestive tract contains various microorganisms, which digest nutrients that cannot be broken down in other parts of the digestive tract. These healthy bacterica can also produce vitamins (K and some B), this is all relient on the balance of ``good'' and ``bad'' bacteria. If this balance is not maintained we will experice discomfort. These ``good'' bacteria are commonly called probiotics, which help with intestinal health. Live/fermented food will contain these helpful bacteria which are known to promote healthy digestion. Another form of microorganisms are called prebiotics, these are dietary fiber that feed the microorganisms contained within the digestive tract, promoting the growth of ``good'' bacteria.\\
\subsubsection{Accessory Organs of Digestion}
\label{sec:orgd3abcee}
The pancreas produces lipase and amylase to break down fats and carbohydrates respectively. Proteases are responsible for breaking down protein. The pancreas also produces bicarbonate to neutralize the stomachs HCl protecting the small and large intestine.\\
\subsubsection{Degestive Disorders}
\label{sec:org4fcbebb}
\textbf{Heartburn} is the condition in which the LES becomes ``loose'' allowing the flow of stomach acid back up into the esophagus. The pain a person experiences is due to the HCl burning the lower esophagus, which lacks the same mucus based protection that the stomach possess. Being overweight may lead to increased heartburn, along with hormone irregularities during pregnancy. If heartburn occurs on a consistent basis it is diagnosed as \textbf{Gastroesophageal Refulx Disease (GERD)}, in which heartburn may occur several times a week. Some of the suspected causes of GERD are: hiatal hernia, obesity, smoking, lack of exercise, medications, stress, and certain foods. To treat GERD it is important to eliminate or fix the cause rather than mediciating symptoms.\\
\textbf{Peptic Ulcers} (stomach ulcers) are the results of erosion in the stomach - producing a hole withing the stomach. This can be caused by HCl or pepsin. This mostly occurs when stomach mucus is worn down due to many causes, leading to erosion of the stomach lining.\\
\textbf{Constipation} occurs when bowles movements occurs too infrequently for a person to be comfortable. When feces takes too long to pass through the intenstines a large amount of water will be absorbed leading to a dry hard stool. Consuming enough fiber, water, and performing frequent exercise are known to relieve symptoms on constipation. Frequent use of laxitaves will cause the bowels to become reliant of the drug.\\
\textbf{Hemorrhoids} are described as sowllen veins in the rectum and anus caused by intense pressure and strain due to constipation. They can also be cause by sitting for too long. Proper fiber consumption and intake of fluids are known to relieve symptoms of hemorrhoids.\\
\textbf{Irritable Bowel Syndrome (IBS)} is a condition more frequent in women than men. Cramping, bloating, and diarrhea/constipation are common symptoms. Causes can include certain foods or stress. Try removing foods that may be causing addition bowel stress.\\
\textbf{Diarrhea} was mainfest as symptoms of cramping; loose, eatery stools; pain; and bloating. It can be caused by a reaction to a specific food, or through bacteria or virus. Diarrhea will cause a lot of eater to exit the body so it is important to treat for dehydration. People experiencing diarrhea should also seek to introduce extra probiotics into their diet.\\
\textbf{Gallstones} are a crystal-like particle that forms from bile and cholesterol. Being overweight, genetic factors, diet, lifestyle, chronic conditions, and supplement use can all lead to gallstones. The stones will cause pain can must be either surgically removed or dissolved by medication.\\
\textbf{Celiac Disease} is a condition in which the body is unable to properly process the protein glueten. Genetic conditions cause the body to experience inflammation within the small intestine causing it to attack itself. Druing the attack villi are permenantly destroyed leading to decreased absorption of nutrients though the small intestine.\\
\section{Chapter 5: Carbohydrates}
\label{sec:org2ec059b}
\subsection{Introduction}
\label{sec:orgddf0578}
There has always been a facination around the roles of carbs in diet, from low carb diets, and most proplr overconsuming carbs. Although the amount of carbs does play a large role in diet and health, the more important factor is the type of carbs you choose to eat. For example whole grains are often neglected in exchange for more of the simple sugars. Each type of carbohydrate has its own role, some provide easy energy for the body, and some like fiber promote gastrointestinal health. We often over consume sugar through both sugar sweetened beverages and snacks.\\
\subsection{Types of Carabohydrates}
\label{sec:orgfdc43bd}
Carbohydrates are made up of carbon, hydrogen, and oxygen. For plant made carbohydrates the energy is captured from the sun, and for animal made carbohydrates from the plants. Mono and Disaccharides make up the class of simple sugars, while polysaccharies make up complex carbohydrates. Starch, glycogen, and dietary fiber are complex carbohydrates. In addition sugar alcohols are similar in structure to polysaccharides, but contain 1-3 calories, as opposed to 4 of normal carbohydrates.\\
\subsubsection{Simple Sugars: Monosaccharides}
\label{sec:orgbfc140d}
The most simple sugars are monosaccharides, that require almost no digestions, and thus are easily absorbed into the blood stream.\\
\textbf{Glucose} is used as the body for its main source of energy, often called blood sugar providing energy to our cells.\\
\textbf{Fructose} is a sugar coming from fruit.\\
\textbf{Galactose} is a sugar found in milk only in combination with glucose.\\
\subsubsection{Simple Sugars: Disaccharides}
\label{sec:orgc7755f4}
Like the name suggests disaccharides consist of 2 sugar molecules, requiring a small amount of digestion to be absorbed by the body.\\
\textbf{Sucrose} is a combination of glucose and fructose, called table sugar. This one molecules can make up around 25 percent of a western diet.\\
\textbf{Maltose} called malt sugar, is made up of two glucose molecules. The molecules is a result of starches being broken during digestion.\\
\textbf{Lactose} is made up od one glucose molecule and one galactose molecule, referred to as milk sugar. Mammals are the only beings capable of producing lactose, it helps onfants absorb calcium, and gain good intestinal bacteria.\\
\subsubsection{Oligosaccharies}
\label{sec:orgd938b56}
This smaller category of carbohydrates are made up of 3-10 monosaccharides. They are found in beans and legumes, they are not digested but instead metabolized in the colon.\\
\subsubsection{Complex Carbohydrates: Polysaccharides}
\label{sec:orga1489f3}
Made up of long glucose chains, these molecules can slowly provide energy or serve as fiber.\\
\subsubsection{Polysaccharides: Starch}
\label{sec:orgbcd7834}
Starch found in plants is made up of branched or unbranched chains of glucose. The branched variety is called amylopectin, while the unbranched variety is called amylose. Amylose has a gel like consistency, while amylopectin is more like wax. Plants will have different ratios of these starches giving them distinct qualities. Starch is commonly used as a thinkening agent.\\
\subsubsection{Polysaccharides: Glycogen}
\label{sec:orgf27042a}
Animals are able to store carbohydrates in the body by using glycogen. After the body gets glucose from foods it can store this glucose as glycogen, its structure is similar to that of plant starch, with a branched chain. 3/4 of a pound of glycogen can be stored in the muscles and liver. Glycogen storage can be manipulated through intake of carbohydrates.\\
\subsubsection{Polysaccharides: Fietary Fiber}
\label{sec:org5f772ac}
Fiber makes up the structure of plants, consisting of long chains of polysaccharides. Although we cannot digest this fiber it helps to keep our gut healthy.\\
Types of Fiber\\
\begin{center}
\begin{tabular}{ll}
Type & Health Effects\\
\hline
Soluble & Lower Cholesterol,  slow glucose absorption\\
Insoluble & Regulate Bowels\\
\end{tabular}
\end{center}
\subsection{Carbohydrate Summary}
\label{sec:org7e733a2}
\begin{center}
\begin{tabular}{llll}
Name & Makeup & Examples & Special Issues\\
\hline
Monosaccharides & One unit & Glucose, Fructose, Galactose & Rarely found alone\\
Disaccharides & Two units & Sucrose, Lactose, Maltose & None\\
OS & 3-10 Units & Raffinose, Stachyose & Causes Gas\\
Polysaccharides & Many Units & Starch, Glycogen & > Glucose\\
Lignin & Many Units & Dietary Fiber & Cannot Digest, Fermented\\
\end{tabular}
\end{center}
\subsection{What Are the Primary Souces of Carbohydrates?}
\label{sec:orge8574a1}
Many types of food contain carbohydrates. Foods with starches or many simple sugars will greatly impact your blood sugar, while nonstarchy vegetables will have very little impact. Many of the foods we consume today contain added sugars, highlighting a problem with our view on food, substituting added sugars for removed fats. In order to get our RDA for carbohydrates we should seek to consume 45-65 percent of our total calories as carabohydrates. Additionally around 130 grams of carbohydrates should be consumed a day to spare protein and avoid ketosis. We should try and eat more whole grain foods as they are nutrient-rich and help maintain feelings of satiety for longer.\\
\subsection{Sugar and Sugar Alternatives}
\label{sec:orgbd76c09}
\subsubsection{Nutritive Sweeteners}
\label{sec:org007ad20}
These are sweeteners that provide energy, such as refined sugar, HFCS, and many more.\\
\begin{center}
\begin{tabular}{ll}
Type & Properties\\
\hline
Sucrose & Table sugar\\
HFCS & Cheap, long lasting\\
Brown Sugar & Moist for cooking\\
Turbinado Sugar & Raw, larger crystals\\
Maple Sugar & Sap from maple tree\\
Honey & Plant nectar + Enzymes\\
\end{tabular}
\end{center}
\subsubsection{Honey}
\label{sec:orgc3dfb68}
Honey is made up of glucose and fructose, but they are not bonded as in sucrose. Honey is not healthier than table sugaar, and it digests just as quickly.\\
\subsubsection{Stevia}
\label{sec:org76e3343}
Stevia is a naturally occuring sweetener that contains very little energy. Has no negative health effects on humans.\\
\subsubsection{Sugar Alcohol}
\label{sec:org5bdd552}
Common names will end in -ol, indicating one molecules of sugar bonded to one molecule of alcohol. Not fully digested by the body.\\
\subsubsection{Nonnutritive Sweeteners}
\label{sec:org7396d3a}
These are commonly called artificial sweeteners, and contain no energy. Aspartame, and sucralose are some examples.\\
\begin{enumerate}
\item Saccharin
\label{sec:org7c73108}
Was thought to cause cancer but has been considered safe since 2000.\\
\item Aspartame
\label{sec:org4420818}
When consumed the body converts aspertame to aspartic acid, phenylalanine, and methanol. Due to the phenylalanine individuals with phenylketonuria need to avoid aspertame. It is not heat stable, so should not be used in cooking. One would need to drink 14 cans of soda to reach dangerous levels.\\
\item Acesulfame-K (Potassium)
\label{sec:orgb048f84}
Considered safe.\\
\item Neotame
\label{sec:orgf306fb0}
7000-13000 times sweeter than sugar, is considered safe.\\
\item Sucralose
\label{sec:org86f4dd7}
Known as splenda, 600 times sweeter than sugar, not digested by the body.\\
\end{enumerate}
\subsubsection{Sweetener Comparison}
\label{sec:org18b0ad0}
\begin{center}
\begin{tabular}{lrl}
Type & Realtive Sweetness & Sources\\
\hline
Lactose & 0.2 & Dairy\\
Maltose & 0.4 & Sprouted seeds\\
Glucose & 0.7 & Corn syrup\\
Sucrose & 1 & Table Sugar\\
Fructose & 1.7 & Fruits\\
Sorbitol & 0.6 & Sugarfree desserts\\
Mannitol & 0.7 & Sugarfree desserts\\
Xylitol & 0.9 & Sugarfree desserts\\
Stevia & 100-300 & Desserts\\
Aspartame & 180-200 & Diet Drinks\\
Acesulfame-K & 150-200 & Chewing gum, desserts\\
Saccharin & 300 & Diet Drinks\\
Sucralose & 600 & Diet Drinks\\
Neotame & 7000-13000 & General Sweetener\\
\end{tabular}
\end{center}
\subsection{Major Functions of Carbohydrates}
\label{sec:org30b91a6}
The most important function of carbohydrates in to provide energy.\\
\subsubsection{Dietary Energy}
\label{sec:orgca077cf}
Certain cells can only gain energy from glucose, which makes sense because most carbohydrates are converted to glucose. An important thing to note is that the speed at which we digest carbohydrates determines how fast we get energy from them, simple sugars digesting very quickly.\\
\subsubsection{Protein Sparing}
\label{sec:org71b99af}
If carbohydrates are under-consumed the body will instead look to other sources of energy. If restricted enough the body will pull energy from muscles, which is detrimental to proper bodily function. Otherwise protein and amino acids will be used for growth and repair. When protein is broken down for energy it is called gluconeogenesis, where glucose is formed from noncarbohydrate sources.\\
\subsubsection{Preventing Ketosis}
\label{sec:org295ecbb}
When energy is needed it can be pulled from fat, where the liver will convert fat into ketone bodies. These are an acidic fat derivative that are formed from an incomplete breakdown of fat. Ketosis is anot a ideal condition for the body and will impair proper function.\\
\subsection{Carbohydrate Digestion}
\label{sec:org1e8affb}
Cooking carbohydrates with high fiber contents will help to soften the tissue making it easier to digest. When ingested salivery amylase will begin to break down starches into much shorter chains. In the stomach carbohydrates do not digest much further, instead they are mainly digested in the small intestine. Pancreatic amylase is used along with maltase, sucrase, and lactase. Once broken down into monosaccharides they can be absorbed in the small intestine. They are then transported to the liver, where they are transformed into glucose and then pushed into the bloodsteam. If the bloodstream already contains adequate glucose levels the liver will produce glycogen, otherwise they will be converted into fat. Fiber will be fermented in the colon later on.\\
\subsection{Complications with Carbohydrate Digestion}
\label{sec:org32b6957}
\subsubsection{Lactose Intolerance}
\label{sec:org371d1e2}
This condition occurs when the body cannot produce lactase. The three types of lactose maldigestion are primary, secondary, and lactose intolerance. Primary is when lactose > lactase, secondary is when the body is temporarly not producing enough lactase. Lactose intolerance is when lactose is fermented producing lactic acid and gas. Often accompanied by diarrhea due to excess water being drawn into the GI tract. One possible way to mitigate the effects of eating lactose would be to consume the lactose with a fat source that would slow digestion, potentially allowing the small amount of lactase present to do its job.\\
\subsubsection{Celiac Disease}
\label{sec:org2bbeb61}
In the US a lot of grains are consumed, and many of the grains provide fiber and protein. A protein called gluten can cause digestion problems for some people, called celiac disease, a condition an inability to absorb nutrients. The villi and microvilli destory themselves rendering them useless at absorbing nutrients. Previously it was thought that 1/6000 people had it, no it is 1/100-200. Better testing, along with different grain processing may be the cause of the sudden increase. It has also become a popular diet so that may explain the increase in ``gluten sensitive'' people.\\
\subsection{Regulating Blood Glucose Levels}
\label{sec:org8942a32}
As discussed the liver helps regulate glucose levels in the blood. When sugar enters the bloodstream insulin is released by the pancreas. \textbf{Insulin} is a ``gatekeeper'', allowing glucose to enter cells, and be used as energy. Resulting in a decrease of glucose in the blood. Insulin also promotes protein synthesis, and coversion of excess glucose to glycogen and fat. Insulin also helps ensure the gluconeogenesis does not occur. Insulin acts to decrease blood glucose levels.\\
When blood glucose levels begin to drop other hormones are released, glucagon, and epinephrine. Glucagon from the pancreas, and epinephrine from the adrenal glands cause glycogen to be released into the blood as glucose. These two hormones also enhance gluconeogenesis, the transformation of fat and protein into blood glucose. These two work together to increase blood glucose.\\
Consuming large amounts of simple sugars on their own will cause a surge in serum insulin, leding to a sharp drop in blood glucose, which can leave a person feeling dizzy and faint.\\
Normal fasting blood glucose levels are considered to be 70-99mg/dL, when above this range the condition is called hyperglycemia (> 125 mg/dL), and below this range being called hypoglycemia (< 40-50 mg/dL).\\
\subsection{Glycemic Response}
\label{sec:org62d311c}
While some carbohydrates may seem ``bad'' and others ``good'' it is important to know that no one carbohydrate is good or bad. They do however have different effects on our blood glucose levels. The \textbf{glycemic} index of a food will tell you how fast it raises a persons blood glucose levels. Glycemic load will be a better indicator as it factors in both rate, and quantity.\\
Effects of Foods with High Glycemic Load\\
\begin{center}
\begin{tabular}{l}
Large release of insulin\\
Increase blood tryglyceride levels\\
Increase fat deposits\\
Increase clotting\\
Increase fat synthesis in liver\\
Rapid return of hunger\\
Developing insulin resustance\\
\end{tabular}
\end{center}
Some may use the glycemic index of foods as a weight loss tool, although it is not proven is research to be a factor in losing weight.\\
\subsection{Health Benefits of Dietary Fiber}
\label{sec:orge0673af}
Soluble fibers are known to lower cholesterol and slow glucose absorption, while insoluble fibers help regular bowel movements. Dietary fiber has many health benefits and is recommened to be part of a healthy diet every single day.\\
\subsubsection{How Much Fiber Do We Need?}
\label{sec:orgc2a9baf}
Experts recommend 25g for women and 38g for men, around 14g/1000cal.\\
\subsubsection{Can You Consume Too Much Fiber?}
\label{sec:org90bb9ee}
Too much fiber can bind up minerals and increase the frequency of bowel movements. In addition too much fiber can take away from consuming carbohydrates containing energy leading to a tired feeling.\\
\subsubsection{To Supplement or Not?}
\label{sec:orga74352b}
\begin{enumerate}
\item Is supplemental fiber in pill or powder form the same as fiber found in food?
\label{sec:org2a81ba4}
Fiber in food contains antioxidants, while in pill form it may not. Additionally fiber pills may dissolve in the stomach leading to digestive issues. These pills can also form a dependence on them in order to facillitiate regular bowel movements.\\
\end{enumerate}
\subsection{Your Health and Carbohydrates}
\label{sec:org5e55fdc}
\subsubsection{Carbohydrate Intake and Obesity}
\label{sec:orgaf94c61}
Although added sugars, and increased fat intake can be a potential cause of obesity, the overarching cause is simply excess calories.\\
\subsubsection{Low-Carbohydrate Diets and Weight Loss}
\label{sec:org2568675}
Low-Carbohydrate diets work on the principle that ketosis sheds fat, and body water. These diets are harsh and not shown to be sustainable in studies. Low-Carbohydrate diets can cause complications with increased fat intake, bad breath, constipation, and dehydration from water loss.\\
\subsection{Diabetes Mellitus}
\label{sec:org2baeb45}
This is a condition in which the body is unable to regulate its blood glucose levels, leading to hyperglycemia. Around 30 million Americans have this form of diabetes.\\
\subsection{Forms of Diabetes}
\label{sec:org10833c8}
\subsubsection{Type 1 Diabetes}
\label{sec:org164c2a6}
The body is unable to produce insulin (thought to be a result of autoimmune disease, and the pancreas being destroyed), as a result a person needs to supplement with insulin.\\
\subsubsection{Type 2 Diabetes}
\label{sec:org2a37fe8}
Also known as noninsulin dependent diabetes mellitus. About 90 percent of all diabetes mellitus is a result of type 2 diabetes. Although many factors can lead to a condition of type 2 disbetes even just a 5-7 percent weight loss can significantly improve insulin sensitivity. Certian genertic makeups have prediposition to this condition.\\
\subsubsection{Gestational Diabetes}
\label{sec:orgf2b47f8}
This type of condition can often occur during pregnancy, patients must work with doctors to make sure the fetus is nat harmed. Mothers that experience this condition are much more likely to get type 2 disbetes later on in life.\\
\subsection{Complications from Diabetes}
\label{sec:orgada7011}
Prolonged hyperglycemia can lead to conditions of ketosis and acidosis. Ketosis occurs when the body can no longer rely on glucose enetering the cells for energy, so the body pulls energy from fat, making ketones. Acidosis is caused by a buildup of acids, both of these conditions combined will lead to a person becoming drowsy and letargic. Additionally the body will also break down proteins for energy, leading to excess kidney stress. Resulting in kindey damage and eventually kindey failure.\\
While there are many risks associated with being diabetes the most deaths occur from cardiovascular disease.\\
Prolonged diabetes will lead to eye damage, blood veins narrowing, nerve damage, never ending infections.\\
While people with diabetes often experice hyperglycemia, they can also experience hypoglycemia from excess insulin usage. Hypoglycemia results in an individual feeling light headed, sick, and eventually becoming unconscious.\\
\subsection{Risk Factors for Type 3 Diabetes}
\label{sec:org1b6acd4}
Although genetic factors can play a role in the onset of diabetes, it is much more important for focus on the factors you can control.\\
Risk Factors\\
\begin{center}
\begin{tabular}{l}
Overweight/Obese\\
> 45 y/o\\
Family History\\
African American\\
High BP\\
Low HDL, High Trigs\\
History of Gestational Diabetes\\
Not active\\
History of Stroke/Heart Disease\\
Depression\\
PCOS\\
Acanthosis Nigricans\\
\end{tabular}
\end{center}
\subsection{Diagnosis of Diabetes}
\label{sec:orgb1d90f2}
If anyone has symptoms they should be tested. Fasting plasma glucose or hemoglobin A1C tests are used.\\
\subsection{Lowering Risks and Treatment for Type 2 Diabetes}
\label{sec:org9b04d5f}
The longer a person has diabetes the worse off they will be. So they should seek to eat less food, and become more active. Control carabohydrates, Increase activity, Take medications to encourage proper insulin response. Try to lower your weight, control glucose intake (make smart choices), better lifestyle choices to help blood pressure and cholesterol, engage in preventative care. Once you have diabetes it will never go away for the rest of your life.\\
\subsection{Dietary Factors to Consider for Diabetes}
\label{sec:org300a3f4}
It is important for a diabetic to eat on a regular pattern in order to ensure proper interaction with medication. The most important energy source to evenly space is carbohydrates. Foucs on heart-healthy food. It is important to get all types of carbohydrates throughout the day, but know foods with starch or simple sugars will have the greatest effect on blood sugar. A carbohydrate counting diet may be implemented to help an individual watch their consumption.\\
\section{Chapter 6: What Are Fats and Lipids?}
\label{sec:org6637ce1}
\subsection{Introduction}
\label{sec:org50bbace}
Fats are lipids that contain nine calories per gram and come in both fat and oil form.\\
\subsection{Forms of Lipids}
\label{sec:org93e6b6e}
Lipids are defined as a substance that cannot dissolve in water. They come in forms such as:\\
\begin{center}
\begin{tabular}{l}
Fatty acids\\
Trigycerides\\
Phospholipids\\
Sterols (cholesterol)\\
\end{tabular}
\end{center}
Trigycerides and phospholipids contain fatty acids.\\
\subsection{Fatty Acids}
\label{sec:org49d7863}
Fatty acids consist of of a chain of carbons linked together. One end is hydrophobic and the other end is hydrophilic.\\
\subsubsection{Fatty Acid Saturation}
\label{sec:orgef85aa8}
\textbf{Saturated Fats} have a hydrogen at every possible position, meaning all carbon carbon bonds are single bonds.\\
\textbf{Unsaturated Fats} have one or more points of unsaturation, where a hydrogen is missing, leading to carbon carbon double bonds. One = mono, two or more = poly.\\
Saturated fats are solid at room temperature and are mainly derived from animal sources. Unsaturated fats are generally liquid at room temperature, and ae derived from plant sources. Foods actually contain a mixture of the three types, but the main type determines its state at room temperature.\\
Foods that contain saturated fats can raise blood cholesterol in some people.\\
\subsection{Types of Lipids}
\label{sec:org4fdf126}
\subsubsection{Triglycerides}
\label{sec:orgc726447}
These make up about 95\% of all lipids, and are what the body can use for energy and storage. They are called triglycerides because of the structure: one glycerol molecules and three fatty acids. The fatty acids can be either essential fatty acids or nonesential fatty acids.\\
\subsubsection{Phospholipids}
\label{sec:org69c6782}
The same structure of a triglyceride with one fatty acid replaced by a phosphate group, giving the molecules emulsification properties.\\
Lecithin is a common emulsifier used in the food industry. Phospholipids allow water and fat interactions within the bodies cells and blood.\\
Phospholipids are also used in the synthesis of acetylcholine.\\
\subsubsection{Sterols/Cholesterol}
\label{sec:org92bfdb3}
Sterols and Cholesterol have a ring like structure as opposed to the chains from above. Many hormones in the body are made from cholesterol, and it is also used in making bile (for food processing in our GI tract).\\
Although cholesterol is a necessary part of bodily function our livers can actually make all that we need. Saturated fat consumption plays a much larger role in serum cholsterol than dietary cholsterol consumption.\\
\subsection{Hydrogenated Fats}
\label{sec:org4061add}
Since saturated fats have some hydrogens missing they are susceptible to oxidation, where they can combine with oxygen. This leads to a oil going rancid. For some oils they will be partially hydrogenated in order to increase product shelf life.\\
The hydrogenatin of oils may lead to the production of trans fatty acids, which is associated with heart disease.\\
The consumption of trans fats can lead to increasing LDL and lowering HDL.\\
Trans fats are now banned from being added to food. It is estimated that half of the trans fats Americans were consuming were from partially hydrogenated oils. Some foods can naturally contain trans fats, these are still allowed.\\
\subsubsection{Butter vs Margarine}
\label{sec:org8d16c72}
The nonhydrogenated spreads are the best option, containing almost no saturated fats, and no cholesterol. New spreads contain emulsifiers to allow for better texture and less fat. Some even contain soy (benecol) which has been shown to increase HDL and lower LDL.\\
\subsection{Lipid Disestion and Absorption}
\label{sec:orge67b3c2}
Fats being hydrophobic face an interesting challenge in the digestion process, they need to work with the water soluble fat enzymes. In order to accomplish this fats need to be broken down to their smallest parts.\\
Salivary, gastric, and pancreatic lipase work to break down fats, going from triglycerides to monoglycerides. Bile acid is also used to emulsify fats into the watery digestive juices.\\
Phospholipids are browken down using pancreatic enzymes, ending up as glycerol, fatty acids, and phosphorus-containing parts.\\
Cholesterol is digested into free cholesterol and fatty acids which are then transported into the blood.\\
\subsection{Lipid Transportation}
\label{sec:orge90a374}
Lipoproteins are used in the transportation of fat, the four types are:\\
\begin{center}
\begin{tabular}{l}
Chylomicrons\\
Very low-density lipoproteins\\
Low-density lipoproteins\\
High-density lipoproteins\\
\end{tabular}
\end{center}
\textbf{Chylomicrons} are formed when fats are absorbed in the small intestine, and they transport lipids to the liver.\\
\textbf{VLDLs} are made in the liver and contain both triglycerides and cholesterol. They become LDLs afterwards.\\
\textbf{LDLs} are produced from VLDLs and will deliver cholesterol to other tissues.\\
\textbf{HDLs} are made in the liver and small intestine. They decrease the risk of heart disease by removing cholesterol from cells and blood vessels, returning it to the liver for breakdown.\\
\subsubsection{Good Cholesterol vs Bad Cholesterol}
\label{sec:org3b2a580}
HDL acts like a cholesterol trash collector for our bodies reducing the risk of heart disease, and clogged arteries.\\
LDL is a cholesterol litter bug and can lead to heart attack, or myocardial infarction.\\
Premenopausal women have high HDL while post menopausal women have low HDL leading to higher risk for heart disease.\\
\subsection{Lipid Metabolism}
\label{sec:orgb9df946}
Fatty acids can either be used for fuel or tuned into triglycerides and stored for later. In muscle they are used to make ATP, and the remaining componets are stored as fat.\\
\subsubsection{Feasting}
\label{sec:org737dbc5}
When excess intake occurs triglycerides are stored in adipose tissue. Adipose tissue has the enzyme lipoprotein lipase on its surface, working to gather triglycerides from lipoproteins. When this excess intake comes from fats it comes directly here via chylomicrons, otherwise protien and carbohydrates must first stop at the liver to be converted into fatty acids, then tryglycerides.\\
\subsubsection{Fasting}
\label{sec:orgb4b0eb0}
When fasting the body will call upon lipase to break down triglycerides into fatty acids and glycerol, which can then be used for energy via ATP.\\
\subsection{Essential Fatty Acids}
\label{sec:orgb39e579}
Essential fatty acids cannot be made by the body, they are: omerga-6 and omega-3.\\
\subsubsection{Omega-3 and Omega-6 Fatty Acids}
\label{sec:org1a87a49}
The 6 in omega-6 refers to the location of the C=C, omega-6 fatty acids make up cell membranes and help blood flow. Omega-3 fatty acids are also cell membranes, and in addition they help prevent inflammation, heart disease, and blood clots. Americans dont get enough omega-3.\\
We can get omega-6 fatty acids from all sorts of plant based oils, while omega-3 most commonly come from soybean, and canola oils, walnuts, and flaxseed. Certain important omega-3 fatty acids are common in cold-water fish. Consumption of these fish has been linked to reduced risk of cardiovascular disease.\\
\subsection{Whats Hot?}
\label{sec:org2b9334a}
\subsubsection{Something fishy -- omega-3 fatty acids}
\label{sec:org9bcd16d}
Studies have shown all sorts of positive health benefits associated with consumption of these essential fatty acids. However excessive supplementation can lead to thinnig of the blood, and lessen the ability fo the blood to clot when it needs to.\\
An adult should consume 3g of omega-3 fatty acids, about 4 servings of fatty fish a week.\\
\begin{center}
\begin{tabular}{ll}
w/o heart disease & Fish 2x/week\\
w/ heart disease & 1g EPA/DHA a day\\
Lowering blood trigs & 2-4g EPA/DHA a day\\
\end{tabular}
\end{center}
\begin{enumerate}
\item Deficiency
\label{sec:orgaf7a921}
\begin{center}
\begin{tabular}{l}
Flaky, itchy skin\\
Diarrhea\\
Risk of infection\\
Stunted growth\\
Reduced wound healing\\
\end{tabular}
\end{center}
\end{enumerate}
\subsection{Functions of Fatty Acids}
\label{sec:org1519d11}
Energy storage, supply essential fats, absorbing vitamins, protecting organs, provide flavor, satiety, cell structure, precursor to steroid hormones\\
\subsubsection{Storage Form of Energy}
\label{sec:org0779ff8}
Fat stores in the average male can total up to around 100000 calories, compared to 1500 on glycogen.\\
\subsubsection{Supply of Essential Fatty Acids}
\label{sec:org52a4180}
This gives the body the fats it cannot produce on its own in the necessary quantities. They are converted into eicosanoids, which have powerful physiologiacal effects including relaxing blood vessels, and promoting clotting.\\
\subsubsection{Absorption and Transport of Fat-Soluble Vitamins}
\label{sec:org5f11ed8}
Vitamins A, D, E, and K\\
\subsubsection{Organ Insulation and Protection}
\label{sec:org63b214c}
Fat surrounds our most important organs, and without it they would not recieve proper insulation and protection. Females have more fat then males. When someone is very lean their body will grow excess hair to try and make up for the lack of insulation (lanugo).\\
\subsubsection{Flavor and Satiety of Food}
\label{sec:orgea39f37}
Fat helps mouthfeel (texture) and the length of a flavors presence. Fats contribute to longer feelings of satiety due to their slower digestion, this also helps flavors stick around for longer. Fat can help those that are dieting to keep them feeling full longer. It will help slow gastric emptying which keeps someone feeling full for longer.\\
\subsubsection{Cell Membrane Structure}
\label{sec:org565a7ea}
Phospholipids make up cell membranes, and membranes of organells. They serve as a barier and allow certain compounds to enter/exit.\\
\subsubsection{Precursor to Steroid Hormones}
\label{sec:org1b2977a}
Sex hormones, aldosterone, and Vit D are all made from cholesterol.\\
\subsection{Recommendation for Dietary Fats}
\label{sec:orgf029c9f}
Adults 20\% not more than 35\%, children 30-35\% no less than 30. Under 2 y/o should not have fats limited.\\
Try to choose unsaturated fats, limit saturated fat intake to 10\% of daily calories.\\
It is important to look at nutrition facts to see ``invisible fats'' as opposed to the fats we can see on meats.\\
\subsection{Risks and Prevention of Heart Disease}
\label{sec:orgc6f2fff}
\subsubsection{Heart Disease}
\label{sec:org50e46db}
Atherosclerosis is a disease that affects both the heart and blood vessels. Plaque build up limits the elaticity of arteries leading to restricted blood flow. High LDL and low HDL are a risk factor along with: smoking, hypertension, high fat intake, family history, diabetes, high homocysteine levels, obesity, physical inactivity, and gender (male higher risk).\\
To reduce risk try to consume more unsaturated fats and less saturated fats. Also look to consumer more fiber.\\
\subsubsection{Cholesterol}
\label{sec:orgda50017}
Hypercholesterolemia (high serum cholesterol), increases ones risk of cardiovascular disease. In some individuals their liver continues to produce cholesterol even when dietary consuption of cholesterol is high, leading to an excess amount of cholesterol in the blood. Try to consume foods lower in cholesterol and saturated fats to lower these levels. Although reducing saturated fat intake is more important than reducing dietary cholesterol. Consuming soy protien and fiber is also recommended as a method for reducing serum cholesterol.\\
\subsubsection{Saturated Fats and Trans Fatty Acids}
\label{sec:orgdf6fc4f}
Saturated fat consumption leads to elevated LDL, increasing the risk of heart disease. Trans fats (now banned) raise LDL and lower HDL, now banned for the increased risk of heart disease.\\
\begin{enumerate}
\item Types of Heart Disease
\label{sec:org12f56bd}
\begin{center}
\begin{tabular}{ll}
Disease & Description\\
\hline
Atherosclerosis & Deposits of fatty substances in arteries\\
Ischemia & Heart blood flow restricted\\
Angina pectoris & Pain in the pectoris from lack of blood flow\\
Myocardial Infarct & Lack of blood flow destroys the heart\\
Stroke & Brain blood flow interrupted\\
\end{tabular}
\end{center}
\end{enumerate}
\subsubsection{Risk Factors for Heart Disease}
\label{sec:org261b306}
Obesiety, blood cholesterol at higher than 200mg/dl. The AHA recommendeds a HDL:Total Cholesterol ratio of 3.5:1 and no higher than 5:1.\\
\begin{center}
\begin{tabular}{ll}
Total Cholesterol & Category\\
\hline
< 200 mg/dl & Desirable\\
200-239 & Borderline High\\
> 240 & Highl\\
 & \\
LDL (Bad) Level & Category\\
\hline
< 100 & Optimal\\
100-129 & Near/Above Optimal\\
130-159 & Borderline High\\
160-189 & High\\
> 190 & Very High\\
 & \\
HDL (Good) Level & Category\\
\hline
< 40 & Risk for HD\\
40-59 & Higher = Better\\
> 60 & Protetive from HD\\
\end{tabular}
\end{center}
Additional risk factors:\\
\begin{center}
\begin{tabular}{l}
Family History\\
High BP\\
High Trigs\\
Smoking\\
Diabetes\\
Obesity\\
Inactivity\\
\end{tabular}
\end{center}
\subsubsection{Signs of a Heart Attack}
\label{sec:org39d4137}
Symptoms can present differently in men a women.\\
Statin medication help lower LDL, and subsequently eliminate cholesterol from the body. Additional medication can be used to reduce trigs.\\
Stints or bypass can also help improve blood flow to the heart.\\
\subsubsection{How Can YOU Decrease Your Risk of Heart Disease?}
\label{sec:orga39cbf9}
\begin{center}
\begin{tabular}{l}
Control BP\\
Keep Cholestrol and Trigs low\\
Maintain a healthy weight\\
Eat healthy\\
Regular exercise\\
Limit Alcohol\\
Dont Smoke\\
Manage Stress\\
Manage Diabetes\\
Sleep enough\\
\end{tabular}
\end{center}
\begin{enumerate}
\item How to lower LDL and Trigs, while raising HDL?
\label{sec:org42027b7}
Reduce Sat fat and cholesterol intake, increase unsaturated fats, increase fiber.\\
Increase antioxidant intake, prevent atherosclerosis by reducing oxidation to LDL molecules which would lead to plaque.\\
Trigs are the most responsive to diet changes:\\
\begin{center}
\begin{tabular}{l}
Avoid overeating\\
Limit alcohol\\
Limit added sugars\\
Included omega-3\\
\end{tabular}
\end{center}
How to raide HDL:\\
\begin{center}
\begin{tabular}{l}
Physical Activity (45 min)\\
Stop Smoking\\
Eat small meals\\
Eat less fat\\
Moderate alcohol intake\\
\end{tabular}
\end{center}
\end{enumerate}
\section{Chapter 7: What is Protein?}
\label{sec:org275a42e}
\subsection{Introduction}
\label{sec:org2c8dad2}
Proteins are the only macronutrient to contain nitrogen, and they can mostly be synthesized by the body. They may act as enzymes, structural components, or the structure causing muscle contraction. Even the bloodstream contains proteins.\\
\subsection{Proteins}
\label{sec:orga7f46b6}
Protiens come from both animal and plant sources. Around 70\% of Americas protein intake comes from animals, as opposed to other places in the world where only 35\% of protein intake comes from animals.\\
Protein can be found in almost every food group.\\
Beef, poultry, milk, white bread, and cheese make up the majority of Americas protein intake. White bread does not contain much protein but we consume so much of it that it makes up a large part of our total protein intake.\\
Unlike carbohydrates and lipids, proteins contain nitrogen, which gives protein its valuable properties. Structure of cells, organs, tissues, regulating nutrient processing, transportation of molecules, and the production of energy.\\
Like the glucose chains of complex carbohydrates, and the fatty acid chains of lipids, proteins are made up of chains of amino acids. These 20 building blocks make up all proteins, each with a different structure.\\
The two types of amino acids are essential and nonessential. The essential amino acids must be consumed in the diet because the body cannot produce enough, while the nonessential amino acids can be produced in the body through a process known as transamination. As long as the body has the necessary building blocks like nitrogen, carbon, hydrogen, and oxygen.\\
There are also conditionally essential amino acids, which are required in certain circumstances, such as disease, or when the essential acids cannot be found in great enough quantity to synthsize the nonessential amino acids.\\
\begin{center}
\begin{tabular}{ll}
Essential & Nonessential\\
\hline
Histidine & Alanine\\
Isoleucine & Arginine\\
Leucine & Asparagine\\
Lysine & Aspartic acid\\
Methionine & Cysteine\\
Phenylalanine & Glutamic Acid\\
Threonine & Glytamine\\
Tryptophan & Glycine\\
Valine & Proline\\
 & Serine\\
 & Tyrosine\\
\end{tabular}
\end{center}
\subsection{From Amino Acids to Proteins}
\label{sec:org6e1b3f6}
Proteins are made of amino acids, and can be broken down and reformed by the body as needed.\\
When amino acids are linked they form a peptide bond, creating a dipeptide, and eventually a polypeptide (aka protein) is formed.\\
The order and shape of a protein detemines its function within the body.\\
While the process of protein production is the same within all cells, each cell will only produce the proteins it needs to function. This process starts from the DNA which contains the ``blueprint'' for how each protein needs to be formed. Cells will produce more or less of these proteins depending on conditions within the body telling them when and how much or each protein to produce.\\
\subsubsection{What happens if the order of an amino acid gets messed up?}
\label{sec:org50e1a7c}
Even just one amino acid being missing can cause a dramatic change to the function of a protein. Take for example \textbf{sickle cell anemia} in which an amino acid is missing from the protin hemoglobin. This causes the shape of the cell to change drastically from a disk to a sickle, causing blockages in the blood vessels, and eventually leading to tissue damage from reduced blood flow.\\
\subsubsection{Denaturation}
\label{sec:org6af98d7}
Denaturation is the process in which a protein's structure (and therefore its function) is changed by heat, acid, enzymes, agitation, or alcohol. Cooking an egg turns it from a liquid mess to a solid shape. Stomach acid performs a very similar role as it denatures proteins we consume.\\
Denaturing a protein increases its surface area allowing for more interaction with digestive enzymes, helping digestion. Stomach acid can denature the proteins of harmful bacteria essentially acting as a second safeguard from infection (the first being cooking of foods).\\
\subsection{Food Sources of Protein}
\label{sec:orgfed4af7}
\subsubsection{Protein Quality}
\label{sec:org6ec2e1e}
\textbf{Complete proteins} contain all essential amino acids, and are easily absorbed (90\%). These are normaly animal proteins, while plant proteins may not contain all essential amino acids or have poor absorption (60\%).\\
\textbf{Incomplete proteins} do not contain all essential amino acids needed by the body. These may be beans, legumes, grains, and vegetables.\\
When a person is not getting the adequate levels of essential amino acids, their body will cease to synthesize any proteins. This can occur when just one essential amino acid is missing, this amino acid being named the ``limiting protein''.\\
When choosing what foods to eat for protein content, it is important to consider other sources of protein out side of animal products. The term \textbf{complementary proteins} refers to two or more incomplete proteins that when eaten together make a complete protein.\\
\begin{center}
\begin{tabular}{lll}
Food & Limited Amino Acid & Complement\\
\hline
Beans & Methionine & Grains, nuts, seeds\\
Grains & Lysine, threinine & Legumes\\
Nuts/seeds & Lysine & Legumes\\
Vegetables & Methionine & Grains, nuts, seeds\\
Corn & Tryptophan, lysine & Legumes\\
\end{tabular}
\end{center}
Since proteins will hang around in the body for some time, it is best to consume these complementary proteins within the same day.\\
These plant sources of protein are much higher in fiber, lower in saturated fat, have antioxidant properties, beneficial mineral content.\\
\subsubsection{Protein Digestion and Absorption}
\label{sec:org22bd919}
Denaturating by HCl allows the enzyme pepsin to digest the peptide bonds of the proteins. Once in the small intestine protease enzymes work to break down polypeptides into short amino acid chains. Which are then absorbed by mucosal cells.\\
\subsubsection{Protein Turnover}
\label{sec:orgf9c955d}
\textbf{Protein Turnover} refers to the constant building (\textbf{protein synthesis}) and destruction (\textbf{protein breakdown}) of prteins withing our cells. The body needs to process in order to grow, repair, and function at ideal rates.\\
The rates of breakdown and synthesis in the body are determined by both body signals, and the amount of protein that is consumed. For example if enough protein is consumed the body is able to properly synthesize muscle, otherwise muscle will be broken down.\\
All essential amino acids must be present in order for protein synthesis to occur, if an amino acid is not available the body may break down muscle to obtain it.\\
Protein has no storage form within the body, so amino acids need to be regularly consumed to keep the \textbf{amino acid pool} topped up. This amino acid pool refers to all free amino acids that are in body fluids. They are waiting to be used for making other proteins, and by the end of the day if they are not used they will either be converted to fat, eliminated as waste, or formed into other nitrogen containing compounds. This waste may be urea in the liver, excreted by the kidneys.\\
\subsection{Protein's Function in the Body}
\label{sec:orgf78c75a}
\begin{center}
\begin{tabular}{l}
Hormones\\
Growth\\
Maintenance\\
Repair\\
Body Structure\\
Blood\\
Fluid Balance\\
Transport\\
Energy\\
Acid Base Balance\\
Immune Function\\
Enzymes\\
\end{tabular}
\end{center}
\subsubsection{Growth, Maintenance, and Repair}
\label{sec:orgf7add25}
The body is contantly in a dynamic state, so consistent protein consumption is necessary to all the turnover processes to occur. The essential amino acids need to be consumed in order to allowt this to occur. Blood cells live 3-4 months, skin days, intestine 3 days.\\
\subsubsection{Body Structure}
\label{sec:org60dbce0}
Protein allows blood to clot, forms ligaments, and forms tendons. \textbf{Collagen} is the most abundent protein in the body making up: tendons, bones, skin, teeth, and blood vessels.\\
\subsubsection{Fluid Balance}
\label{sec:org196fe5c}
\textbf{Albumin} circulating in the blood helps keep a balance of fluids in cells and outside of cells. When protein intake is insufficient fluids can easily enter the cells leading to \textbf{edema}.\\
\subsubsection{Acid-Base Balance}
\label{sec:orgc39afc1}
Proteins act as a pH buffer, to neutralize the pH.\\
\subsubsection{Immune Function}
\label{sec:orgd736902}
\textbf{Antigens} are foreign substances that threaten the body's health. \textbf{Antibodies} are the proteins that the immune system produces as a defense against these antigens. The musuc in our respiratory systems and intetines are made up of amino acids which is contantly secreted to trap bacteria before it can ever enter the body. Without proper protein intake the immune system is weaker and will not be able to keep you healthy.\\
\subsubsection{Enzymes}
\label{sec:org39d0677}
\textbf{Enzymes} are proteins that speed up chemical reactions. They consist of 100-1000 amino acids and help all throughout the body. In cetrain genetic abnormalitites they may not be produced. This is known as an inborn error of metabolism. One example is \textbf{hemophilia} in which excessive bleeding and bruising can occur from minor injuries. In these people the genetic code for clotting does not exist. Injecting the body with clotting aids can help these people.\\
\subsubsection{Hormones}
\label{sec:org6dccec2}
\textbf{Hormones} are the bodies messenger. Some are lipids and some are proteins.\\
\subsubsection{Transport}
\label{sec:org90f4a2f}
Proteins that transport substances across cell walls act as pumps. Another transport protein within the cell is the sodium-potassium pump. They help maintain the balance of sodium and potassium on the inside and outside of a cell. Proteins can also transport nutrients in the bloodstream. Albumin makes up 60\% of the bodies plasma and it is responsible for the transport of drugs and thyroid hormones, plus is carries fatty acids from adipose tissue to muscle cells for use as energy. Proteins all fat and water to mix.\\
\subsubsection{Energy}
\label{sec:org36f77d3}
Through gluconeogenesis the body can use protein as an energy source, but during this process protein cannot performs its regular functions as listed above. It is not the best energy source for the body.\\
\subsubsection{Recommendations for Protein Intake}
\label{sec:orgde8d6f0}
0.8g per kilogram of body weight. Around 10\% of caloies for the day. If you are injuried or an athlete you may need more.\\
\subsubsection{Myths and Legends}
\label{sec:org5d6b3c2}
More protein is required for athletes, notably when beginning a new exericise program.\\
Athletes may want to consume 1.2-1.4g of protein per kilogram of body weight, or 1.6-1.7g for strength athletes. There is no need to go crazy with protein consumption.\\
\subsection{Protein Balance}
\label{sec:orga49fb27}
When amino acids are broken down nitrogen is released, and this nitrogen can be measured thorugh the protein we eat, or the nitrogen we excrete. Most of the time this in and out is balanced, but during times of growth, recovery, pregnancy, and muscle growth the body is in a \textbf{positive nitrogen balance}, when intake is greater than excretion. In the opposite situation you would have \textbf{negative nitrogen balance}, this may be due to excess breakdown due to AIDS, cancer, starvation, or restictive diets. Most people are balanced, with US consuming too much protein.\\
\subsubsection{Too Little Protein}
\label{sec:orga538267}
If someone is protein deficient it is likely that protein is not the only nutrient they are deficient in. \textbf{Protein energy malnutrition (PEM)} occurs when someone is not gettng enough protein, calories, or both. It is rare but may occur when a person is wasting, or malnourished. PEM is the most lethal form of malnutrition.\\
\begin{enumerate}
\item Marasmus and Kwashiorkor
\label{sec:orgdff29d2}
People with \textbf{Marasmus} will be in an emaciated state, or have a skeletal apperance due to inadequate intake of \textbf{protein \emph{and} calories}. Can occur with improper nutrient absorption, or self-starvation.\\
\textbf{Kwashiorkor} can affect children being weaned from breastfeeding. The transition from high protein breast milk to low protein cereal will cause edema in the belly area. Essentially under consuming of protein.\\
\end{enumerate}
\subsubsection{Too Much Protein}
\label{sec:org9faa14c}
Americans often consume too much protein without the need for the high consumption. What are the consequences?\\
High-protein diets may lack the necessary vegetables for fiber. They will also often be high in saturated fats. Generally animal based diets will be low in fiber, phytochemicals, vitamins, and minerals.\\
High-protein diets may lead to an increased risk for cancer and heart disease. Notably colon, prostate, breast, and pancreatic cancers. If calcium is underconsumed bone loss can also occur.\\
High-protein diets may also cause additional stress to the kidneys, which may lead to kidney disease. This is not a concern for the average healthy person, but for infants if their is not enough water intake there may be additional stress on the kidneys. 2g/kg seems to be the upper limt for benefits.\\
\subsection{Vegetatianism}
\label{sec:orgcc3146f}
\subsubsection{Why Do People Choose to Become Vegetarians?}
\label{sec:org2078031}
Sometimes meat is cost prohibitive, and it is restricted in some religions.\\
\subsubsection{Different Types of Vegetarian Diets}
\label{sec:org7313acc}
\begin{center}
\begin{tabular}{ll}
Vegans & Omit all animal products\\
Lacto-vegetarians & Include dairy products\\
Lacto-ovo-vegetarians & Incluse eggs and dairy\\
Lacto-ovo-pesce vegetarians & Include eggs, dairy, and seafood\\
Semi-vegetarians & Sometimes eat meat/seafood\\
\end{tabular}
\end{center}
\subsubsection{Health Benefits of a Vegetarian Diet}
\label{sec:orgef50257}
Lower risk of heart disease, prevention of some cancers, lower BMI, less likely to be obese. Eating a variety of plant foods is recommended for all the various micronutrient benefits associated with their consumption.\\
\subsubsection{Health Challenges for Vegetarians}
\label{sec:orgbc8b44e}
Nutrient imbalance or deficiency. Low iron, zinc, calcium, protein, B6, B12.\\
\section{Chapter 8: What is Alcohol?}
\label{sec:orgbfb7ac8}
\subsection{Introduction}
\label{sec:org96e5273}
Alcohol is consumed for socializing, celebrations, relaxation, escape, or as port of cultural and religious practices. Moderate consumption can have health benefits, but only 50\% of people consume alcohol in a moderate amount.\\
\subsection{What is Alcohol?}
\label{sec:org9816d40}
Also known as ethanol, alcohol is a intoxicating ingredient in beer, wine, and liquor. It is not a nutrient, and is not required for health. It does provide energy at 7cal/gram, and has properties that allow it to dissolve lipids, and is made through carbohydrate fermentation. Yeast breaksdown sugars into alcohol, CO2, and water.\\
\textbf{Alcohol Proof} is used to label beverages by alcohol content. Proof is twice the percentage of alcohol by volume, 100-proof whiskey is 50\% alcohol by volume. Beer is 4-5\%, wine is 7-15\%, and spirits are 40-50\%.\\
\subsection{Metabolism of Alcohol}
\label{sec:org33922ad}
No digestion is required, instead alcohol can be directly absorbed into the bloodstream through diffusion. 1-3\% of alcohol is peed out, and 1-5\% evaporates via breath. Everything else must go through the liver. Only 1/5 of the alcohol is absorbed through the stomach, with the rest occuring in the small intestine. The enzymes \textbf{alcohol dehydrogenase} and \textbf{acetaldehyde dehydrogenase} are responsible for the metabolism of alcohol.\\
Alcohol has no storage form in the body, thus the body needs to process it right away, first it is converted to acetaldehyde (via alcohol dehydrogenase) and then acetate (via acetaldehyde dehydrogenase). After being spread through the bloodstream it can be used as energy in the liver and other places in the body. In excess amounts alcohol will destroy the lipid membrane of cells. Additionally all the excess energy in the liver will be converted to fat leading to fatty liver.\\
\subsection{Factors that Affect Intoxication}
\label{sec:orgc83f244}
\subsubsection{Gender}
\label{sec:orgedf3105}
Women will experience a faster rise in BAC due to: lower body water (leading to more alcohol per unit of body water), less muscle, and more fat. They also metabolize alcohol slower due to having less alcohol dehydrogenase. The effects of alcohol can also hang around longer in women.\\
\subsubsection{Race/Ethnicity}
\label{sec:org4b2e48d}
Asians experience flushing, possibly due to a genetic mutation surrounding acetaldehyde dehydrogenase.\\
\subsubsection{Body Size}
\label{sec:orge78a46c}
Smaller people will feel effects fast, and will metabolize alcohol slower.\\
\subsubsection{Amount of lacohol and other foods consumed (before or during alcohol intake)}
\label{sec:orgd39cca2}
Some foods may biffer the absorption or metabolism of alcohol.\\
\subsubsection{Sleep}
\label{sec:orgb09a775}
Sleep can impact metabolism, affecting alcohol metabolism.\\
\subsubsection{Medications}
\label{sec:orgc8e33c7}
Alcohol can lessen the effectiveness of certain medication, or may cause issues.\\
\subsubsection{Other Factors}
\label{sec:org1d56f2d}
Genetics, organ efficiency, or other issues will influence alcohol metabolism. It has calories but is not a good energy source.\\
\subsection{Metabolic Effects of Excessive Alcohol}
\label{sec:org1044bfe}
Can have detrimental effects of both organs and body systems. These include the brain, liver, cardiovascular system, digestive system, and immune system. Chronic alcoholism will lead to nutrient deficiency due to the lack of actual nutrients present in alcohol. May also cause vomiting which removes nutrients from the body.\\
Excess urination can also occur from alcohol consumption, which will lead to a drop in mineral content of the body, leading to typical symptoms like cramps and worse.\\
\textbf{Wernicke-Korsakoff} syndrome is a brain disorder caused by lack of B-vitamin thiamin, common among alcoholics. Other vitamins such as A, D, E, K, C, B (thiamin, niacin, folate, B6, B12) will also have decreased levels in the body due to excess alcohol consumption. Minerals such as calcium, phosphorus, potassium, magnesium, zinc, and iron may also have their levels in the body lessened.\\
Digestive organs can also be damaged from alcohol consumption. The GI mucus lining will be damaged due to excess stomach acid and histamine. This may trigger heartburn and ulcers. Alcohol can also lessen digestive enzyme secretion, along with the absorption of nutrients.\\
Due to its effects on the liver bile production will be lessened, impacting the bodies ability to digest fats, and the fat-soluble vitamins. With all the excess energy in the body fat synthesis will be increased.\\
\textbf{Alcoholic Fatty Liver Disease} is caused by excess alcohol use. The metabolism/breakdown of alcohol primarially occurs in the liver, which can damage the liver, promote inflammation, and weaken the body's natural defenses. This condition can be reversed with abstinence from alcohol. With continued consumption comes \textbf{alcoholic hepatitis} which may include symptoms of pain, diarrhea, nausea, and vomiting. Lastly \textbf{cirrhosis} can occur where accumulation of fat causes hepatocytes (liver cells) to burst and die. They are replaced by scar tissue that cannot function like the liver tissue it is replacing, which will lead to a non functioning liver.\\
\subsection{Defining Moderate Intake}
\label{sec:org493f9ea}
\subsubsection{So how much is too much?}
\label{sec:org08534f4}
Consumption is linked to car crashes, violence, sexual risk behaviors, high BP, and various cancers. These risks can occur at very low levels of alcohol consumption. This misuse of alcohol is most common among young adults, and they are at the most risk for alcohol's acute effects. Starting at a young age can also form a pattern of drinking that leads to potential future alcoholism.\\
Binge drinking is defined as having a patern of you BAC over 0.08g/dl or above. >5 drinks in men, >4 drinks in women, in a 2 hour period. Most binge drinkers do not have a severe alcohol use disorder.\\
\begin{center}
\begin{tabular}{rr}
Age Range & Binge Percentage\\
>18 & 17\%\\
18-24 & 25\\
25-34 & 26\\
35-44 & 19\\
45-64 & 14\\
65+ & 4\\
\end{tabular}
\end{center}
Men should strive for less than 2 drinks a day, while women should go for less than 1 drink. If you don't already drink dont start.\\
One ``Drink'' can be defined as:\\
\begin{center}
\begin{tabular}{l}
12 oz Beer\\
5 oz Wine\\
1.5 oz Spirit\\
\end{tabular}
\end{center}
Each of these servings contains approximately 15 grams of alcohol, or 105 calories.\\
\subsection{Potential Benefits of Alcohol Use}
\label{sec:org560bda9}
\begin{center}
\begin{tabular}{l}
Possibly\\
\hline
Reduce Heart Disease\\
Reduce Ischemic Stroke\\
Reduce Diabetes\\
\end{tabular}
\end{center}
Other factors such as diet and exericise have astronomically more benefits to reducing risk of disease as compared to alcohol consumption. Even small drinkes are at much higher risk for certain cancers.\\
\subsection{Who Should Absolutely Avoid Alcohol?}
\label{sec:org0901ade}
\subsubsection{Pregnancy Women \& Fetal Alcohol Syndrome}
\label{sec:org9d9b2c2}
No safe amount of alcohol during pregnancy. Consumed alcohol will make its way to the fetus causing higher BAC in the fetus than the mother. This can interrupt the delivery of oxygen and nutrients to the fetus. This will harm the development of tissues and organs, and potentially cause permanent brain damage.\\
\textbf{Fetal Alcohol Syndrome} occurs when a fetus is exposed to alcohol during pregnancy. It can cause brain damage and growth problems. Adverse outcomes linked with Fetal Alcohol Syndrome inclue:\\
\begin{center}
\begin{tabular}{ll}
Physical Defects & CNS Abnormalities\\
\hline
Slow Growth & Off balance\\
Reduced head size & Dumb/Bad memory\\
Extremity malformation & Delayed mental growth\\
Organ malfunction & Bad logic\\
Eye/Ear impaired & ADD ADHD\\
Delay Fine/Gross motor & Temper\\
\end{tabular}
\end{center}
\section{Chapter 9: What Are Vitamins?}
\label{sec:org9aff28b}
\subsection{Vitamins: A Little Goes a Long Way}
\label{sec:org0e7f516}
We only need very small amounts of micronutrients, measured in milligram and microgram amounts. While micronutrients are not used for bodily structure they are vital to powering the chemical reaction that make our body tick. Vitamins are carbon containing (organic) while minerals are not. The body contains pounds of both carbohydrates and fat, while only containing less than a teaspoon of vital nutrients such as B1.\\
\subsection{Fat-Soluble versus Water-Soluble Vitamins}
\label{sec:orgd6d45ba}
\begin{center}
\begin{tabular}{l|ll}
 & Water-Soluble & Fat-Solbule\\
\hline
Absorption & Direct > Circulation & Lymph > Ciculation\\
Transport & Unbound & Transport Protein\\
Storage & In water-filled areas & In fat-storing cell\\
Excertion & Filtration, kidney>urine & Remains in fat\\
Toxicity & Not common, supplements & Supplements\\
Frequency & Daily or EOD & Week or Month\\
\end{tabular}
\end{center}
Vitamin Naming Breakdown\\
\begin{center}
\begin{tabular}{ll}
Fat-Soluble & Water Soluble\\
\hline
Vit A (Beta-Carotene, Retinol) & Vit B1 Thiamin\\
Vit D & Vit B2 Riboflavin\\
Vit E & Vit B3 Niacin\\
Vit K & Vit B5 Pantothenic Acid\\
 & Vit B6 Pyridoxine\\
 & Vit B7 Biotin\\
 & Vit B9 Folate\\
 & Vit B12 Cobalamin\\
 & Vit C Ascorbic Acid\\
\end{tabular}
\end{center}
\subsubsection{Body Regulation}
\label{sec:orge5003fe}
For most nutrients their bioavailibilty (absorption \% of nutrient) is increase when they are consumed in foods rather than their simple form. This is due to the presense of other nutrients in other foods. The vitamin D added milk helps absorption of calcium. Bioabailibility can range from 40-90\% due to factors such as:\\
\begin{center}
\begin{tabular}{l}
Efficiency of digestion\\
Previous nutrient intake\\
Foods consumed together\\
Method of preperation\\
Source of nutrient\\
\end{tabular}
\end{center}
\subsubsection{Preserving Your Vitamin Intake}
\label{sec:org53dd59b}
Vitamins and Minerals easily be destroyed in the cooking process, or even during transportation and storage.\\
\begin{center}
\begin{tabular}{l}
Reduce amount of water used in cooking\\
Reduce cooking time\\
Reduce surface area\\
\end{tabular}
\end{center}
Boiling is the worst, while microwaving and steaming are the best options for vegetables. Frozen vegetables are often the same or better than fresh vegetables. Canned foods can have excess losses due to the canning procedure.\\
\begin{center}
\begin{tabular}{l}
Keep foods cool to prevent enzymes destroying vitamins\\
Keep foods cool, moist, and air-tight\\
Keep foods in bigger pieces\\
Minimize reheating\\
Dont add baking soda\\
Store canned foods correctly\\
Eat the canned food liquid\\
\end{tabular}
\end{center}
\subsection{Fat-Soluble Vitamins}
\label{sec:org3f53d94}
These are stored in fat, so no need to over-consume them because they stick around.\\
\begin{center}
\begin{tabular}{ll}
Vitamin A & Vision, Cell differentiation, Hormone\\
Vitamin D & Bones, Calcium, Immunity, Hormone\\
Vitamin E & Antioxidant, Skin, Immunity\\
Vitamin K & Clotting, Bone health\\
\end{tabular}
\end{center}
\subsubsection{Vitamin A}
\label{sec:org79fd637}
\textbf{Retinol, Retinal, and Retinoic Acid} are the forms of Vitamin A in the body. This group is called \textbf{retinoids}. Also found in plants in the form of \textbf{carotenoids}, the most important of which being \textbf{beta-carotene}.\\
Vit A is sent to the liver, then into the blood with a retinol-building protein. \textbf{Retinol Activity Equivalents (RAE)} are units to tell use how much of a certain nutrient is needed to have the equilivent power of one mcg of retinol.\\
\begin{center}
\begin{tabular}{l}
Vision\\
Gene Regulation\\
Immune Function\\
Bone Growth\\
Reproduction\\
Cell Membrane\\
Opithelial Cells\\
Cotrisol Synth\\
Nerve Insulation\\
Production Blood\\
Thyroid Hormone\\
\end{tabular}
\end{center}
Vitamin A deficiency is the leading cause for non-accidental blindness worldwide. Vit A also helps keep the cornea moist and healthy, if defiecient keratin-forming cells with proliferate and the eye will dry out, becoming susceptible to xerophthalmia. Bitot spot can also form.\\
Macular Degenerative Disease can be slowed or prevented through proper Vit A intake. The center field of vision becomes dark and clouded.\\
Espicially in children Vit A is essential for immune function.\\
Gene production for metabolism is regulated by Vit A.\\
Vit A signals bone cells to break down the inner part of a bone, deficiency may lead to stunted childhood growth.\\
Epithelial tissue can be compromised. Vit A also helps cells grow mature are differentiation, if deficient this can lead to cells dying early and being replaced by cells of different function.\\
Toxicity will happen in the same tissue where deficiency can occur. Symptoms include: headaches, abdominal pain, skin rashes, liver damage, diarrhea, nausea, hair loss, joint pain, and bond and muscle soreness. Itching skin, blurred bision, growth failure in children, and increased bone fractures are also common.\\
Pregnancy women should take caution to avoid Via A over-consumption.\\
DRI is 900/700mcg for men and women. Breastfeeding takes these numbers up to 1200-1300. UIL is at 3000mcg/day.\\
\subsubsection{Food Sources of Vitamin A}
\label{sec:orgd52ea0a}
Dairy, liver, eggs, margarine, spinach, dark greens, and broccoli. Orange and red foods also have high Vit A.\\
\subsubsection{Vitamin D}
\label{sec:orgf3c0ed7}
Also known as cholecalciferol, is additionally classified as a hormone, and steroid. It can be made from cholesterol. It controls bone development and maintenance, through metabolism of calcium and phosphorus. It stimulates cells in the small intestine to increase calcium absorption. It works to keep serum calcium levels in the right zone.\\
\textbf{Rickets} is what bowed legs is called, caused by a Vit D deficiency
\end{document}
