% Created 2022-01-27 Thu 10:19
% Intended LaTeX compiler: pdflatex
\documentclass[11pt]{article}
\usepackage[utf8]{inputenc}
\usepackage[T1]{fontenc}
\usepackage{graphicx}
\usepackage{longtable}
\usepackage{wrapfig}
\usepackage{rotating}
\usepackage[normalem]{ulem}
\usepackage{amsmath}
\usepackage{amssymb}
\usepackage{capt-of}
\usepackage{hyperref}
\author{Jackson}
\date{\today}
\title{Notes}
\hypersetup{
 pdfauthor={Jackson},
 pdftitle={Notes},
 pdfkeywords={},
 pdfsubject={},
 pdfcreator={Emacs 27.2 (Org mode 9.6)}, 
 pdflang={English}}
\begin{document}

\maketitle
\tableofcontents


\section{Day 1 syllabus}
\label{sec:org80f12fa}
\subsection{{\bfseries\sffamily TODO} add dates from syllabus to calendar}
\label{sec:orga57d153}
\subsection{Grading}
\label{sec:org7fe52fa}
\subsubsection{Exams 80\%}
\label{sec:org06f6e90}
\subsubsection{labs 20\%}
\label{sec:org1c875d7}
\subsection{Inroduction and Chemical Biology}
\label{sec:org01621e2}
\begin{center}
\begin{tabular}{lr}
Element & Symbol\\
\hline
H & 63\%\\
O & 26\\
C & 9\\
N & 1\\
\end{tabular}
\end{center}
\begin{itemize}
\item atoms are made up of protons, neutrons, and electrons
\item we only care about outer electons
\end{itemize}
\subsubsection{Molecule vs Compound}
\label{sec:orgeb0e862}
\begin{itemize}
\item two or more atoms vs two or more different elements
\end{itemize}
\subsubsection{Types of bonding}
\label{sec:orgae09e10}
Covalent bonds: electrons are shared among atoms
\begin{itemize}
\item polar: unequal sharing
\item non polar: equal sharing
\item single and double covalent bonds
\begin{itemize}
\item depends on how many electrons are being shared
\end{itemize}
\item Polar examples
\begin{itemize}
\item Oxygen and hydrogen can form a covalent bond
\begin{itemize}
\item water is a polar substance
\end{itemize}
\item Sulfur and hydrogen can also
\item Same with N and H
\end{itemize}
\item Non polar examples
\begin{itemize}
\item C and H
\item C and C
\end{itemize}
\end{itemize}
Ionic bonds: One or more electrons from one atom are removed and attached to another atom, forms cations and anions
\begin{itemize}
\item this will make both atoms ``happy'', or closer to a completed valence shell
\begin{itemize}
\item These can also be called salts
\end{itemize}
\item Examples
\begin{itemize}
\item Na Cl, Na has one, Cl has 7, they can combine and make each complete
\end{itemize}
\end{itemize}
Hydrogen bonds: Weak bonds between hydrogen and other atoms
\begin{itemize}
\item Water is polar and it makes a ton of hydrogen bonds by orienting its O to another water molecules H (weak Hydrogen bond)
\end{itemize}
\begin{center}
\begin{tabular}{llll}
Chemical Atom & Symbol & Ion & Electrons Gained or Lost\\
\hline
Hydrogen & H & Hydrogen Ion & 1 Lost\\
Sodium & Na & Sodium Ion & 1 Lost\\
Potassium & K & Potassium Ion & 1 Lost\\
Chlorine & Cl & Chloride Ion & 1 Gained\\
Magnesium & Mg & Magnesium Ion & 2 Lost\\
Calcium & Ca & Calcium Ion & 2 Lost\\
\end{tabular}
\end{center}
\subsubsection{Electronegativity}
\label{sec:org5ec3bdf}
\begin{itemize}
\item when two atoms with differing electronegativity bond the electrons will concentrate more towards the atom with greater electronegativity (polar covalent bonding)
\begin{itemize}
\item if two atoms with the same electronegativity bond they will form a non polar bond
\item non polar bonds are hydrophobic
\item polor bonds are hydrophilic
\item polar dissolves polar
\end{itemize}
\end{itemize}
\subsubsection{Ionic Molecules}
\label{sec:org7cbcd1a}
\begin{itemize}
\item Ionization
\begin{itemize}
\item carboxyl group
\begin{itemize}
\item (-COOH)
\end{itemize}
\item amino group
\begin{itemize}
\item (-NH2)
\end{itemize}
\end{itemize}
\end{itemize}
\subsubsection{Fucntion Groups}
\label{sec:org3a9d0ef}
\begin{itemize}
\item Ionized Group
\begin{itemize}
\item Carboxyl
\item Amino
\item Phosphate
\end{itemize}
\end{itemize}
\section{Day 2}
\label{sec:org72dc7db}
\subsection{Free Radicals}
\label{sec:orge578488}
\begin{itemize}
\item atoms of molecules containing a single electron
\item Unstable, and highly reactive
\end{itemize}
\subsection{Solutions}
\label{sec:org1c79762}
\begin{itemize}
\item the liquid everything is disolved in is the solvent
\item Everything else is the solute
\item water is a universal solvent
\item however not all molecules can be dissolved in water
\end{itemize}
\subsubsection{rxn with water}
\label{sec:org372e01d}
\begin{itemize}
\item hydrolysis breaking of a chemical bond with the additions of elements of water -H and -OH to the products
\item dehydration involves a removal of water, one net water molecules is removed to combine two small molecules into one larger one
\end{itemize}
\subsection{Osmosis}
\label{sec:org940f0f5}
\begin{itemize}
\item water moving between fluid compartments
\item water moves from areas of low concentrations of solutes, to areas of high solute concentration, essentially creating two equal areas of solute concentration
\item rate of diffusion: 10 billion molecules per second
\end{itemize}
\subsection{Solubility in water}
\label{sec:org7316e74}
\begin{itemize}
\item polar molecules will easily dissolve in water: hydrophilic
\item non polar molecules will not easily dissolve in water: hydrophobic
\end{itemize}
\subsection{Amphipathic Molecules}
\label{sec:orge33f8e9}
\begin{itemize}
\item a special class of molecules that have a polar or ionized region at one site and a nonpolar region at another site
\item in water these molecules will form clusters with their polar regions at the surface of the cluster, and the non polar sites inwards
\item these help dissolve non polar substances in the presense of water
\item plasma membrane structure helps transport molecules in the blood
\end{itemize}
\subsection{Concentration}
\label{sec:org077ad03}
\begin{itemize}
\item the amount of solute present in a unit volume of solution
\item moles/liter is an example
\end{itemize}
\subsection{Acids and Bases}
\label{sec:org20adde6}
\begin{itemize}
\item molecules that release H are called acids
\item mocules that accept H are called bases
\item Hydrogen is very useful in our bodys energy system due to the fact that it is very simple
\item The bodies pH range is around 7.35 to 7.45
\begin{itemize}
\item blood 7.4
\end{itemize}
\end{itemize}
\subsection{Terminology of different Sciences}
\label{sec:org2fa46a7}
\begin{itemize}
\item Organic is C-H
\item Inorganic in non C
\item Biochem is living organisms
\end{itemize}
\subsection{Classes of Organic Molecules}
\label{sec:org0f4b113}
\subsubsection{Carbohydrates}
\label{sec:orgbb823c6}
\begin{itemize}
\item Disaccharides and polyaccharides
\end{itemize}
\subsubsection{Lipids}
\label{sec:org7456cc4}
\begin{itemize}
\item Triglycerides
\item Phospholipids
\item Steroids
\end{itemize}
\subsubsection{Proteins}
\label{sec:org52cebb7}
\begin{itemize}
\item polypeptides
\end{itemize}
\subsubsection{Nucleic Acids}
\label{sec:org74d81d9}
\begin{itemize}
\item DNA and RNA
\end{itemize}
\subsection{Organic chemicals}
\label{sec:orgc4d3b36}
\begin{itemize}
\item compounds containing carbon bonded to hydrogens
\item carbon is the fundamental element of life
\begin{itemize}
\item 4 atoms in valence
\item single, double, triple covalent bonds
\item linear, branched, or ringed molecules
\end{itemize}
\end{itemize}
\subsection{Fucntional Groups Continued}
\label{sec:org2aceb21}
\begin{itemize}
\item smaller groups of atoms that bind to organic compound
\item confer unique reactive properties on the whole molecules
\item Hydroxyl (O-H), found in alcohols, and carbohydrates
\item Carboxyl (COOH), found in fatty acids, proteins, and organic acids
\item Ester (COOR), found in lipids
\item Carbonyl (COH), found in aldehydes, polysaccharides
\item Phosphate (PO4H2), found in DNA, RNA, ATP
\item Methyl (CH3), found in DNA, amino Acids, Lipids, Carbohydrates
\end{itemize}
\subsection{Carbohydrates: Basic Structure}
\label{sec:orgbd03fb7}
\begin{itemize}
\item General Formula (CH2O)
\item Basic Structure:
\begin{itemize}
\item Backbone of Carbon
\item Polyhydroxy aldehyde or ketone
\end{itemize}
\item Common Configurations
\begin{itemize}
\item Monosaccharide: polyhydeoxy aldehyde or ketone with 3-7 carbons
\item Disaccharide: two monosaccharides
\item Polysaccharide: five or more monosaccharides
\end{itemize}
\item Changing the chiral orientation of just one C will change the molecule
\item Different arrangements will create different structual properties:
\begin{itemize}
\item linear: celluose, structural integriety
\item branched: starch, glycogen, storing energy, easy to pull apart and access
\begin{itemize}
\item you can pull whole branches off of the structure for easy energy access
\end{itemize}
\end{itemize}
\item Combining two different monosaccharides will form a new carbohydrate
\begin{itemize}
\item table sugar
\end{itemize}
\end{itemize}
\subsection{How the body uses sugar}
\label{sec:org2e58fb8}
\begin{itemize}
\item glycogen exists in the body as a resevoir of available energy that is stored in the chemical bonds within individual glucose monomers
\item blood sugar
\end{itemize}
\subsection{Lipids}
\label{sec:org839607d}
\begin{itemize}
\item moleucles composed of mostly hydrogen and carbon
\item linked by non polar covalent bonds, they are nonpolar, low solubility in water
\begin{itemize}
\item fatty acids, triglycerides, phospholipids, steroids
\item act as a boundry
\end{itemize}
\item valuable store of energy
\item major component of all cellular membranes
\item important signaling molecules
\end{itemize}
\subsubsection{Fatty acids}
\label{sec:orgd384d04}
\begin{itemize}
\item hydrocarbon chain, and a carboxyl group
\item all single bonds: saturated fatty acid
\item one or more double bonds: unsaturated fatty acids
\item >1: polyunsaturated
\item 1: monounsaturated
\end{itemize}
\subsubsection{Triglycerides}
\label{sec:org405daca}
\begin{itemize}
\item the majority of the lipides in the body
\item glycerol, a three-carbon sugar-alcohol, bonded to three fatty acids
\item present in blood and cn be synthesized in the liver
\item stored in great quantities in adipose tissue
\item energy reserve or the body, during fasting or exercise
\end{itemize}
\subsubsection{Phospholipids}
\label{sec:orgfb3887c}
\begin{itemize}
\item similar in overall sructure to triglycerides, but the third hydroxyl group of glycerol is linked to phosphate
\item a small polar (ionized nitrogen-containing molecule) is usually attached to the phosphate
\item polar region at one end, two fatty acids make a non-polar region at the opposite end
\item they are amphipathic
\begin{itemize}
\item they form lipid bilayers of cellular membranes
\end{itemize}
\end{itemize}
\subsubsection{Steroids}
\label{sec:org19c047f}
\begin{itemize}
\item a distinctly different structure from those of the other subclasses of lipid molecules.
\item four interconnected rings of carbon atoms form the skeleton of every steroid
\item no water-soluble
\item cholesterol, cortisol, estrogen, testosterone
\item cholesterol is inserted into the phospholipid bilayer
\begin{itemize}
\item reinforcing the membrane
\end{itemize}
\end{itemize}
\subsubsection{Proteins}
\label{sec:org03e0d5f}
\begin{itemize}
\item about 50 percent of the organic material in the body (17 percent by weight)
\item carbon, hydrogen, oxygen, nitrogen, and small amounts of sulfur
\item they are macromolecules, thousands of atoms
\item 20 amino acids
\item polymer: peptide, polypeptide, protein
\end{itemize}
\begin{center}
\includegraphics[width=.9\linewidth]{/home/jackson/Pictures/Screenshot_2022-01-27_09-06-18.png}
\end{center}
\begin{itemize}
\item amino acids are attached through peptides bonds to from proteins
\item proteins fold into very specific 3D shapes
\item functions: support, enzymes, transport, defense, movement
\begin{itemize}
\item Primary: a series of amino acids bound in a chain. amine acids display small charges functional groups
\item Secondary: develops CO- and NH- groups on adjacent amino acids form hydrogen bonds. This action folds the chain into local configurations called the alpha helic and the beta pleated sheet. Most proteins have both types of secondary structures
\item Tertiary: portions of the secondary structure further interact by forming covalent disulfide bonds and additional interaction. From this emerges a stable three-dimensional molecule. Dependong on the protein, this may be the final function state.
\item Quaternary: Exists only in proteins that consist of more than one polypeptide chain.
\end{itemize}
\item Two variables detrmine the primary structure of a protein
\begin{itemize}
\item the number of amino acids in the chain
\item the specific sequence of different amino acids
\end{itemize}
\end{itemize}
\subsection{Major Catergories anf Function of Proteins}
\label{sec:orgd09e2ba}
\begin{center}
\begin{tabular}{lll}
Category & Functions & Ex\\
\hline
Proteins that regulate gene expression & make RNA/DNA, make polypeptides from RNA & transcription factors activate genes; RNA polymerase transcribes genes\\
Transporter proteins & Mediate the movement of solutes such as ions and organic molecules across plasma membranes & ion channels in plasms membranes allow movement across the memebrane of ions such and Na1 and K1\\
Enzymes & Accelerate the rate of specific chem rxns, such as those required for cellular metabolism & lipase, amylase, proteases\\
Cell Signaling proteins & Enable cells to communicate with each other, themselbes, and with the external environment & plasma memvrane receptors bind to hormones or neurotransmitters in extracellular fluid\\
Motor proteins & initiate movement & myosin, found in muscle cells, contractile force\\
Structural proteins & support, connect, and strengthen cells, tissues, and organs & collagen\\
Defense proteins & protext against infection and disease & cytokines and antibodies\\
\end{tabular}
\end{center}
\subsection{{\bfseries\sffamily TODO} Take notes of 1-4 protein structures}
\label{sec:orgde4eb31}
\subsubsection{Protein Conformation}
\label{sec:org5691e7e}
\begin{itemize}
\item in nature proteins appear folded,  bended, or twisted forming more compact strctures
\item this is known as a proteins conformation
\end{itemize}
\subsubsection{Primary Structure}
\label{sec:org03c896c}
\begin{itemize}
\item primary structure is determined by:
\begin{itemize}
\item the number of amino acids in the chain
\item the specific sequence of different amino acids
\end{itemize}
\item Kinda like a linear chain, just not an exact straight line
\end{itemize}
\subsubsection{Secondary Structure}
\label{sec:org648f76a}
\begin{itemize}
\item Attraction between various regions along this linear chain form hydrogen bonds and thus create the secondary structure in a protein
\item These are called peptide bonds
\item these bonds occur at regular intervals and force the conformation into a spiral or alpha helix
\item in addition hydrogen bonds can also form between peptide bonds when extended regions of a polypeptide chain run parallel to each other, forming a relatively straight, extended region known as a beta pleated sheet
\item In between these two structures random coil conformations help to link the two together
\item these two structures give the protein its ability to anchor itself into a lipid bilayer
\end{itemize}
\subsubsection{Tertiary Protein Structure}
\label{sec:orgdaed8f0}
\begin{itemize}
\item after secondary structures are formed additional amino acid side chains become possible
\item they fold the polypeptide into three-dimensional conformations,  forming a functional protein
\end{itemize}
\begin{enumerate}
\item Determining Tertiary Structure
\label{sec:orga9916fd}
\begin{enumerate}
\item hydrogen bonds between side groups of amino acids or with surrounding water molecules
\item ionic interactions between ionized regions along the chain
\item interactions between nonpolar regions
\item covalent disulfide bonds linking the sulfur-containing side chains of two cysteine amino acids
\item van der Waals forces
\end{enumerate}
\begin{center}
\includegraphics[width=.9\linewidth]{/home/jackson/Pictures/fold.png}
\end{center}
\end{enumerate}
\subsubsection{Quaternary Protein Structure}
\label{sec:org21dfd37}
\begin{itemize}
\item if more than one polypeptide chain is bonded together it is known as a quaternary structure, or multimeric proteins
\item the same forces act upon these proteins as described above
\item therefore the subnits are held together in the same ways
\end{itemize}
\end{document}
